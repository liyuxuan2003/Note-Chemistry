% !TEX program = xelatex
\documentclass[UTF8]{ctexart}

\RequirePackage{inputenc}
\RequirePackage{fontspec}
\RequirePackage{xeCJK}

\RequirePackage{amsmath}
\RequirePackage{mhchem}
\RequirePackage{chemfig}
\RequirePackage{graphicx}

\RequirePackage{geometry}

\RequirePackage{ruby}

\RequirePackage{tikz}

\usetikzlibrary{shapes}

\setmainfont{Times New Roman}

\setCJKmainfont{等线}
\setCJKsansfont{等线}
\setCJKmonofont{等线}

\let\nvec\vec
\def\vec#1{\nvec{\vphantom b\smash{#1}}}

\newcommand{\rnum}[1]{\uppercase\expandafter{\romannumeral #1\relax}}

\geometry
{
    left=1.25in,
    right=1.25in,
    top=1in,
    bottom=1in
}

\title{化学笔记}
\author{李宇轩}
\date{2019.08.26}

\begin{document}
\maketitle

\newpage

\tableofcontents

\newpage

\setlength{\parindent}{0pt}

% \section{物质结构}

% \subsection{原子结构}
%     物质是由原子组成的,而原子则是由原子核和核外电子组成的,
%     原子核非常小,却集中了原子绝大多数的质量。
%     原子核是由质子和中子组成,质子带正电,中子不带电,电子带负电,
%     原子中质子和电子的数量相等,所以原子呈电中性。\\[3mm]
%     在原子中,电子的质量非常小,可以忽略不计,
%     质子和中子的质量较大,且两者的质量非常接近。
%     为了便于使用,
%     我们规定一个包含$6$个质子和$6$个中子的碳原子质量的$1/12$为相对原子质量的单位。
%     所以我们可以认为,一个原子的相对原子质量在数值上等于其质子数和中子数的和。\\[3mm]
%     具有相同质子数的原子称为元素,而具有相同质子数和中子数的原子称为核素,例如:\\[3mm]
%     原子核中具有六个质子的原子称为碳元素。\\[1mm]
%     原子核中具有六个质子和六个中子的原子称为碳-12核素。\\[1mm]
%     原子核中具有六个质子和七个中子的原子称为碳-13核素。\\[3mm]
%     每一种元素的核素也可以称为这种元素的同位素。
%     显然同一种的元素的不同核素在元素周期表上的位置是相同的,
%     这就是同位素这个名字的来源。\\[3mm]
%     通常我们会这样表示质子数和中子数:
%     \begin{table}[h]
%         \begin{center}
%             \begin{tabular}{l|l}
%                 \hline
%                 质子数符号:$P$\qquad\qquad&质子英文:Proton\qquad\qquad\\ \hline
%                 中子数符号:$N$\qquad\qquad&中子英文:Neutron\qquad\qquad\\ \hline
%             \end{tabular}
%             \caption{质子和中子的符号}
%         \end{center}
%     \end{table}\vspace{-20pt}

% \subsubsection{原子核的稳定性}
%     原子核的稳定性与质子和中子的比值有关,如果某种核素的质子和中子的比值
%     基本符合下方的函数关系时,那么这种核素是基本稳定的,
%     偏离这种关系越多,该种核素越不稳定。\\
%     \begin{equation*}
%         \begin{cases}
%             \ \dfrac{N}{P}=1.0\qquad P\in[1,20]\\[6mm]
%             \ \dfrac{N}{P}=1.0+0.01\cdot(P-20)\qquad P\in[21,83]\\
%         \end{cases}
%     \end{equation*}\\
%     对于原子序数1-20的元素,
%     元素中最稳定的核素的原子核内质子数等于中子数,
%     这种情况下维持原子核稳定的中子和质子的比值始终为$1.0$。\\[3mm]
%     对于原子序数21-83的元素,
%     随着质子数的增多,核内的斥力增大,
%     所以需要多于质子数量的中子以增大核内的引力,从而抵消质子的影响,
%     且随着质子数的增加,会需要更多额外的中子来维持原子核的稳定。
%     这种情况下维持原子核稳定的中子和质子的比值会从$1.0$逐步提高至$1.6$。\\[3mm]
%     对于原子序数83-118的元素,
%     对于这一类元素,不存在的稳定的核素,所有的同位素均不稳定。

% \newpage

% \subsubsection{原子核的衰变}
%     若某种核素不稳定,就代表这种核素会发生核衰变,核衰变分为两种:

% \paragraph{$\alpha$衰变}
%     原子发生$\alpha$衰变的时候,会抛出两个中子,两个质子,两个电子。\\[1mm]
%     $\alpha$衰变会产生一个$\alpha$粒子和两个$\beta$粒子,
%     两种粒子分别以$\alpha$射线和$\beta$射线的形式放出。

% \paragraph{$\beta$衰变}
%     原子发生$\beta$衰变的时候,一个中子会变为一个质子,一个电子,其中电子会被抛出。\\[1mm]
%     $\beta$衰变会产生一个$\beta$粒子,以$\beta$射线的形式放出。\\

%     其中$\alpha$粒子的本质是由两个中子和两个质子结合而成的氦离子,
%     而$\beta$粒子的本质就是电子。\\[3mm]
%     通过以下表格可以对比$\alpha$衰变和$\beta$衰变:\vspace{3pt}
%     \begin{table}[h]
%         \begin{center}
%             \begin{tabular}{l|l|l|l|l}
%                 \hline
%                 衰变类型\qquad&质子数(P)\qquad&中子数(N)\qquad&相对原子质量的变化\qquad&元素周期表上的变化\\ \hline
%                 $\alpha$衰变&$-2$&$-2$&$-4$&向前移动两格\\ \hline
%                 $\beta$衰变&$+1$&$-1$&$~0$&向后移动一格\\ \hline
%             \end{tabular}
%             \caption{$\alpha$衰变和$\beta$衰变的规律}
%         \end{center}
%     \end{table}\vspace{-20pt}

% \subsection{原子核外电子排布}
%     核外电子的排布可以由内向外分为数个电子层,
%     电子层用主量子数$n$编号,$n$越小电子层越靠内,$n$越大电子层越靠外。\\
%     \begin{table}[h]
%         \begin{center}
%             \begin{tabular}{l|l|l|l|l}
%                 \hline
%                 主量子数\qquad\qquad&1\qquad\qquad&2\qquad\qquad&3\qquad\qquad&4\qquad\qquad\\ \hline
%                 电子层符号&K&L&M&N\\ \hline
%             \end{tabular}
%             \caption{主量子数和电子层符号}
%         \end{center}
%     \end{table}\\
%     电子层$n$中包含了$n$个能量不同的电子亚层,
%     电子亚层用副量子数$l$编号,$l$越小电子亚层能量越低,$l$越大电子亚层能量越高。\\
%     \begin{table}[h]
%         \begin{center}
%             \begin{tabular}{l|l|l|l|l}
%                 \hline
%                 副量子数\qquad\qquad&0\qquad\qquad&1\qquad\qquad&2\qquad\qquad&3\qquad\qquad\\ \hline
%                 电子亚层符号&s&p&d&f\\ \hline
%             \end{tabular}
%             \caption{副量子数和电子亚层符号}
%         \end{center}        
%     \end{table}\\
%     如果我们需要表示某个特定电子层的电子亚层,
%     可以将电子层的主量子数加在电子亚层符号前,
%     例如:1s、2s、2p、3s、3p、3d~......
    
% \newpage

%     每一个电子亚层都包含数条轨道,副量子数为$l$的电子亚层有$2l+1$条轨道,
%     而根据泡利不相容原理,每个轨道上最多可以容纳2个电子,
%     所以一个电子亚层可以容纳$4l+2$个电子。\vspace{5pt}
%     \begin{table}[h]
%         \begin{center}
%             \begin{tabular}{l|l|l|l|l}
%                 \hline
%                 电子亚层符号\qquad\qquad&s\qquad\qquad&p\qquad\qquad&d\qquad\qquad&f\qquad\qquad\\ \hline
%                 轨道数量&1&3&5&7\\ \hline
%                 电子数量&2&6&10&14\\ \hline
%             \end{tabular}
%             \caption{电子亚层的轨道数量}
%         \end{center}
%     \end{table}\\
%     每一个电子层所包含的轨道数等于该电子层内$n$个电子亚层轨道数之和,
%     也就是前$n$个奇数的和,或者说是$n$的平方,
%     所以主量子数为$n$的电子层有$n^2$条轨道,
%     而根据泡利不相容原理,每个轨道上最多可以容纳2个电子,
%     所以一个电子层可以容纳$2n^2$个电子。\vspace{5pt}
%     \begin{table}[h]
%         \begin{center}
%             \begin{tabular}{l|l|l|l|l}
%                 \hline
%                 电子层符号\qquad\qquad&K\qquad\qquad&L\qquad\qquad&M\qquad\qquad&N\qquad\qquad\\ \hline
%                 轨道数量&1&4&9&16\\ \hline
%                 电子数量&2&8&18&32\\ \hline
%             \end{tabular}
%             \caption{电子层的轨道数量}
%         \end{center}
%     \end{table}\\
%     \textbf{泡利不相容原理:}每一轨道最多能容纳自旋方向相反的两个电子。\\[6mm]
%     电子填充电子亚层的顺序是由低能级到高能级,或者说电子优先填充能量较低的电子亚层。\\[3mm]
%     我们已经知道在同一个电子层中,副量子数越大,电子亚层的能量越高,
%     而对于不同电子层中电子亚层的能量高低规律,可以总结为下方的表格:\vspace{5pt}
%     \begin{table}[h]
%         \begin{center}
%             \begin{tabular}{l|l|l|l|l}
%                 \hline
%                 电子层$K$&\textbf{(01.)}\quad1s[2e]&&&\\ \hline
%                 电子层$L$&\textbf{(02.)}\quad2s[2e]&\textbf{(03.)}\quad2p[6e]&&\\ \hline
%                 电子层$M$&\textbf{(04.)}\quad3s[2e]&\textbf{(05.)}\quad3p[6e]&\textbf{(07.)}\quad3d[10e]&\\ \hline
%                 电子层$N$&\textbf{(06.)}\quad4s[2e]&\textbf{(08.)}\quad4p[6e]&\textbf{(10.)}\quad4d[10e]&\textbf{(13.)}\quad4f[14e]\\ \hline
%                 电子层$O$&\textbf{(09.)}\quad5s[2e]&\textbf{(11.)}\quad5p[6e]&\textbf{(14.)}\quad5d[10e]&\textbf{(17.)}\quad5f[14e]\\ \hline
%                 电子层$P$&\textbf{(12.)}\quad6s[2e]&\textbf{(15.)}\quad6p[6e]&\textbf{(18.)}\quad6d[10e]&\textbf{(20.)}\quad6f[14e]\\ \hline
%                 电子层$Q$&\textbf{(16.)}\quad7s[2e]&\textbf{(19.)}\quad7p[6e]&\textbf{(21.)}\quad7d[10e]&\textbf{(22.)}\quad7f[14e]\\ \hline
%             \end{tabular}
%             \caption{电子亚层填充顺序}
%         \end{center}
%     \end{table}\\
%     如果需要直接比较两个电子亚层的能量高低,
%     可以首先比较两者主量子数和副量子数的和,数值越大,能量越高,
%     若和相等,则比较两者的主量子数,数值越大,能量越高。\\[6mm]
%     \textbf{洪特规则:}电子在同一个电子亚层中能量相同的轨道上排布时,
%     总是尽可能的分占不同的轨道且自旋方向相同,只有当各轨道均有一个电子时,
%     才允许自旋方向相反的电子进入成对。

% \newpage

% \subsection{元素周期表}
%     元素周期表中的行称为周期,列称为族。\\[3mm]
%     由于有电子亚层的存在,且电子总是优先向能量较低处排布,
%     所以有时是最外层的电子层中的电子数量增加,
%     有时是次外层的电子层中的电子数量增加。\\[3mm]
%     元素周期表中的元素分为主族元素和副族元素,
%     主族元素指的是最外层的电子数量增加的元素,
%     副族元素指的是次外层的电子数量增加的元素。
%     由于元素的性质主要取决于最外层的电子数,
%     所以主族元素中,元素性质的差异较为明显,
%     而副族元素中,元素性质的差异较为微弱。\\[6mm]
%     对于主族元素,我们有以下性质:\\[3mm]
%     同一周期内电子层的数量相等,同一族内最外层的电子数量相等。\\[3mm]
%     对于同一周期中的元素,从左至右,最外层的电子数量递增。\\[3mm]
%     对于同一族的元素,从上至下,电子层的数量递增。\\

% \subsubsection{元素的金属性和非金属性}
%     通过这张表格,我们可以充分认识元素的金属性和非金属性的实际意义。
%     \begin{table}[h]
%         \begin{center}
%             \begin{tabular}{l|l|l|l}
%                 \hline
%                 性质\qquad\qquad&意义\qquad\qquad\qquad\qquad&与“最外层电子数”的关系\qquad\qquad\qquad&与“电子层数”的关系\qquad\qquad\qquad\\ \hline
%                 金属性&失去电子的能力&最外层电子数越少,金属性越强&电子层数越多,金属性越强\\ \hline
%                 非金属性&得到电子的能力&最外层电子数越多,非金属性越强&电子层数越少,非金属性越强\\ \hline
%             \end{tabular}
%             \caption{元素的金属性和非金属性}
%         \end{center}
%     \end{table}\\
%     对于最外层的电子数量,我们可以这样理解:\\[2mm]
%     如果最外层有$1$个电子,元素会倾向于失去$1$个电子以达到稳定,所以此时金属性较强。\\[2mm]
%     如果最外层有$7$个电子,元素会倾向于得到$1$个电子以达到稳定,所以此时非金属性较强。\\[6mm]
%     对于电子层的数量,我们可以这样理解:\\[2mm]
%     如果一个元素的电子层数很少,
%     那么原子核对于最外层的电子的吸引力就较强,
%     所以最外层的电子就很难被夺走,
%     元素会倾向于通过得到电子的方式达到稳定,
%     非金属较强。\\[2mm]
%     如果一个元素的电子层数很多,
%     那么原子核对于最外层的电子的吸引力就较弱,
%     所以最外层的电子就很容易被夺走,
%     元素会倾向于通过失去电子的方式达到稳定,
%     金属性较强。

% \newpage

% \subsubsection{元素周期表中金属性和非金属性的规律}
%     结合主族元素性质的规律,我们可以得到以下表格:
%     \begin{table}[h]
%         \begin{center}
%             \begin{tabular}{l|l|l}
%                 \hline
%                 条件\qquad\qquad\qquad\qquad&金属性\qquad\qquad\qquad\qquad&非金属性\qquad\qquad\qquad\qquad\\ \hline
%                 同一周期的元素(从左至右)\qquad\qquad&金属性减弱&非金属性增强\\ \hline
%                 同一族内的元素(从上至下)\qquad\qquad&金属性增强&非金属性减弱\\ \hline
%             \end{tabular}
%             \caption{元素周期表中金属性和非金属的规律}
%         \end{center}
%     \end{table}\\
%     根据表格中所指出的规律,
%     我们可以推测出,非金属性最强的元素位于右上角,
%     即氟元素(\ce{F}),
%     金属性最强的元素位于左下角,即铯元素(\ce{Cs})。\\[3mm]
%     同时在主对角线上会出现一些金属性和非金属性相当的元素,
%     如铝元素(\ce{Al}),如硅元素(\ce{Si}),
%     这些元素在不同的条件下,既可以表现出金属的特征,
%     也可以表现出非金属的特征。

% \subsection{电负性}
%     电负性衡量了元素在化合物中吸引电子对的能力,
%     元素的电负性越大,在化合物中吸引电子对的能力越强,
%     元素的电负性越小,在化合物中吸引电子对的能力越弱。\\[3mm]
%     通常来说,电负性越大的元素非金属性越强,
%     电负性越小的元素金属性越强。
%     非金属性最强的元素氟的电负性为$3.98$,
%     金属性最强的元素铯的电负性为$0.79$。\\[3mm]
%     电负性小于$1.8$是金属元素。\\[2mm]
%     电负性大于$1.8$是非金属元素。\\[2mm]
%     电负性在$1.8$附近的元素则同时具有金属性和非金属性。

% \subsubsection{电负性和化学键}

%     \textbf{如果原子B的电负性远远大于原子A:}
%     这意味着原子B对电子对的吸引能力远远大于原子A,
%     原子B会夺取电子形成阴离子,
%     原子A会失去电子形成阳离子。\\[2mm]
%     此时两个原子间形成的化学键称为\textbf{离子键}。\\[3mm]

%     \textbf{如果原子B的电负性略微大于原子A:}
%     这意味着原子B对电子对的吸引能力略微大于原子A,
%     电子对会倾向于原子B一侧,但原子B并不能彻底夺取电子,
%     两个原子共用了这一对电子。\\[2mm]
%     此时两个原子间形成的化学键称为\textbf{极性共价键}。\\[3mm]

%     \textbf{如果原子B的电负性等于原子A:}
%     这意味着原子B对电子对的吸引能力与原子A是完全相等,
%     电子对会处在原子A和原子B的中间,
%     两个原子非常平均的共用了这一对电子。\\[2mm]
%     此时两个原子间形成的化学键称为\textbf{非极性共价键}。

% \newpage

% \subsection{离子键}
%     我们一般认为如果两个原子的电负性差值大于$1.7$时,
%     那么这两个原子间的化学键为离子键,
%     所以离子键大多形成于电负性差异较大的金属和非金属间。\\[3mm]
%     离子键的第一个特点是没有方向性,
%     由于离子的电荷分布是球状对称的,
%     离子对每一个方向的静电力都是相同的,
%     所以离子键是没有方向性的。\\[3mm]
%     离子键的第二个特点是没有饱和性,
%     每一个离子都可以同时吸引多个与其电荷相反的离子,
%     可以如此在空间中无限制的延伸下去。\\

% \subsection{共价键}
%     我们一般认为如果两个原子的电负性差值小于$1.7$时,
%     那么这两个原子间的化学键为共价键,
%     所以共价键大多形成于电负性差异较小的两种非金属或两种金属间。\\[3mm]
%     如果两个元素种类相同的原子形成共价键,
%     由于同种元素的电负性相同,
%     两者的电负性差值为零,
%     所以形成的是非极性共价键。\\[3mm]
%     如果两个元素种类不同的原子形成共价键,
%     由于不同元素的电负性即便再接近也必然会有一定差异,
%     所以形成的一定是极性共价键。\\[3mm]
%     共价键的极性强弱衡量了形成共价键的两个原子电负性差值的大小,
%     电负性差值越大,共价键的极性越强,
%     电负性差值越小,共价键的极性越弱,
%     如果一个共价键的极性非常强,
%     那么这个共价键可能就会变为离子键。\\[3mm]
%     共价键的第一个特点是有方向性,
%     由于大部分的电子云在空间中都有一定的伸展方向,
%     为了形成稳定的共价键,
%     只有沿特定的方向才能使电子云最大程度的重叠,\\[3mm]
%     共价键的第二个特点是有饱和性,
%     共价键是通过共用电子的方式成键,
%     显然一个原子不可能无限制的进行电子对的共用,
%     所以共价键只能将有限个的原子组成分子,
%     而分子之间则需要依靠分子间作用力来构成宏观物质,
%     这就是共价键的饱和性。\\

% \subsection{金属键}
%     在金属中,有一部分金属原子会失去它们的价电子,
%     形成对应的金属阳离子。
%     脱离金属原子的自由电子会像气体一般弥散在整个金属中,
%     与失去电子的金属阳离子间产生强烈的静电作用,
%     这种作用力被称为金属键。\\[3mm]
%     金属中除了自由移动的电子,
%     还同时存在原子和离子,而且这两者并非一成不变的,
%     会不断地通过得到和失去电子在彼此之间来回变化,
%     最终达成动态平衡。\\[3mm]
%     金属键只存在于金属中。

% \newpage

% \subsection{晶体}
%     晶体指的是构成晶体的微粒以特定的方式,
%     规则排列而形成的宏观物质。

% \subsubsection{离子晶体}
%     离子晶体是由离子在离子键的作用下构成的,
%     离子键的作用力较强,所以离子晶体的硬度较大,
%     熔沸点较高。\\[3mm]
%     离子键晶体内阴阳离子键密集堆积,
%     相互作用力很强,离子无法自由移动,
%     所以离子晶体并不导电,
%     但是当离子晶体在熔融状态或溶于水时是可以导电的,
%     这是因为在熔融状态下,晶体中的离子键被高温破坏,
%     离子可以自由移动,所以在熔融状态时离子晶体可以导电,
%     而当离子晶体溶于水时,阴阳离子在水的作用下被拆散,
%     离子键被破坏,离子可以自由移动,所以在溶于水时离子晶体也可以导电。\\[3mm]
%     典型的离子晶体:\ce{NaCl}(氯化钠),\ce{CaCO3}(碳酸钙)。

% \subsubsection{原子晶体}
%     原子晶体是由原子在共价键的作用下构成的,
%     共价键的作用力非常强,所以原子晶体硬度非常大,
%     熔沸点非常高。\\[3mm]
%     原子晶体中不存在可以自由移动的电荷,
%     所以通常不导电。\\[3mm]
%     原子晶体可以只由一种原子构成,
%     其中较为典型的有碳单质和硅单质,
%     金刚石是碳单质中唯一符合原子晶体构型的晶体,
%     在金刚石晶体中,每个碳原子都与相邻的四个碳原子形成共价键。\\[3mm]
%     原子晶体也可以由多种原子构成,
%     其中较为典型的有碳化硅,
%     碳化硅在结构上是将金刚石晶体中用硅原子替代一半的碳原子,
%     使得每个碳原子周围存在四个硅原子,
%     而每个硅原子周围存在四个碳原子,
%     均以共价键相互连接。\\[3mm]
%     典型的原子晶体:\ce{C}(碳),\ce{Si}(硅),\ce{SiC}(碳化硅)。

% \subsubsection{分子晶体}
%     分子晶体是由分子在范德华力的作用下构成的,
%     而分子内部是由共价键维持的,
%     范德华力的本质就是分子间作用力,与化学键相比非常弱,
%     所以分子晶体的硬度较小,熔沸点较低,
%     且常温下通常为气体或液体。\\[3mm]
%     分子晶体中,虽然分子间的范德华力很弱,
%     晶体在温度很低时就会熔化,
%     但是由于组成分子晶体的分子仍然是由共价键维持的,
%     所以在熔化时分子本身并不会分解。
%     水在常温下为液体,但是仍然存在水分子,
%     只有当温度超过$1000^{\circ}\text{C}$时
%     才会有极少量的水分子分解形成氢气和氧气。\\[3mm]
%     绝大多数的有机物均为分子晶体。\\[3mm]
%     典型的分子晶体:\ce{O2}(氧气),\ce{Cl2}(氯气),\ce{HCl}(氯化氢),\ce{CH4}(甲烷)。

% \newpage

% \subsubsection{金属晶体}
%     金属晶体是由金属原子在金属键的作用下构成的,
%     金属键与离子键较为相似,
%     均是由于正负电荷间的相互作用而产生,
%     因此金属键的作用力也相对较强,
%     所以金属在常温下大多都是固体,
%     而且非常坚硬。\\[3mm]
%     金属具有四个特性:导电性,
%     导热性,延展性,具有金属光泽。\\[3mm]
%     金属的导电性是由于金属中存在大量可以自由移动的电子。\\[3mm]
%     金属的导热性是由于电子可以通过与附近的原子或离子不断碰撞从而转递热量。\\[3mm]
%     金属的延展性是由于金属键的强度非常大,
%     在极强的外力下仍然不会被破坏。\\[3mm]
%     实际上,延展性包含了延性和展性两个方面,
%     延性指的是可以将金属拉伸为细丝,
%     比如铜丝,铁丝等,展性指的是可以将金属轧为薄片,
%     比如金箔,铝箔等。

% \subsubsection{石墨的特殊晶体结构}
%     石墨是碳单质的另一种形式,
%     具有非常特殊的晶体结构。
%     石墨整体来看为层状结构,
%     每一层中的碳原子分别于三个碳原子通过共价键相连,
%     形成六边形的网状结构,
%     而层和层之间又通过范德华力维持。
%     可以看出石墨同时具备了分子晶体和原子晶体的特点,
%     我们应当将其视为由复杂键型组成的晶体。\\[3mm]
%     石墨中每一层内的碳原子只连接了三个碳原子,
%     存在多余的电子,而这些多余的电子则会成为自由电子,
%     所以石墨是可以导电的。\\[3mm]
%     石墨中每一层内的作用力较强,
%     而层和层之间的作用力较弱,
%     在外力作用下,石墨的层和层之间极易发生相对位移,
%     所以石墨可以起到润滑作用。

% \newpage

% \section{氟~~氯}

% \subsection{氯}
%     氯,黄绿色气体,
%     熔点$-101^\circ$C,
%     沸点$-35^\circ$C。\\[3mm]
%     氯是一种主族元素,氯通常显$-1$价,有时也可以显$+1$价,$+3$价,$+5$价,$+7$价。\\[5mm]
%     氯气可以和金属钠反应,剧烈燃烧,生成白色固体:
%     \begin{center}
%         \ce{2Na + Cl2 ->T[点燃] 2NaCl}
%     \end{center}
%     \vspace{5pt}
%     氯气可以和金属铁反应,剧烈燃烧,生成棕褐色的烟:
%     \begin{center}
%         ~~~~\ce{2Fe + 3Cl2 ->T[点燃] 2FeCl3}
%     \end{center}
%     \vspace{5pt}
%     氯气可以和金属铜反应,剧烈燃烧,生成棕黄色的烟:
%     \begin{center}
%         \ce{Cu + Cl2 ->T[点燃] CuCl2}
%     \end{center}
%     \vspace{10pt}
%     氯气可以和磷反应,生成三氯化磷或五氯化磷:
%     \begin{center}
%         \ce{2P + 3Cl2 ->T[点燃] 2PCl3}\\[3mm]
%         \ce{2P + 5Cl2 ->T[点燃] 2PCl5}\\[3mm]
%     \end{center}
%     氯气和氢气可以在点燃的条件下发生反应,安静的燃烧,生成具有刺激性气味的氯化氢:
%     \begin{center}
%         \ce{H2 + Cl2 ->T[点燃] HCl}
%     \end{center}
%     \vspace{5pt}
%     氯气和氢气可以在光照的条件下发生反应,剧烈的爆炸,生成具有刺激性气味的氯化氢:
%     \begin{center}
%         \ce{H2 + Cl2 ->T[光照] HCl}
%     \end{center}
%     \vspace{5pt}
%     在第一个反应中,氢气和氯气燃烧时的火焰为苍白色。\\[3mm]
%     在第二个反应中,通常我们会用镁带燃烧产生的强光照射氯气和氢气的混合气体,触发反应。\\[5mm]
%     氯气的水溶液称为氯水,氯气可以和水反应,生成盐酸和次氯酸:
%     \begin{center}
%         \ce{Cl + 2H2O <=> HCl + HClO}
%     \end{center}
%     \vspace{5pt}
%     反应生成的次氯酸不稳定,在光照下会分解为盐酸和氧气:
%     \begin{center}
%         \ce{2HClO ->T[光照] 2HCl + O2 ^}
%     \end{center}
%     次氯酸具有强氧化性,可以杀死病菌,
%     而氯气和水反应可以得到次氯酸,所以自来水常用氯气进行消毒,
%     次氯酸也具有漂白性,可以使有机色素褪色。

% \newpage

%     氯气可以和氢氧化钠反应,生成氯化钠和次氯酸钠(以及水):
%     \begin{center}
%         \ce{Cl2 + 2NaOH -> NaCl + NaClO + H2O}
%     \end{center}
%     \vspace{5pt}
%     氯气可以和氢氧化钙反应,生成氯化钙和次氯酸钙(以及水):
%     \begin{center}
%         \ce{2Cl2 + 2Ca(OH)2 -> CaCl2 + Ca(ClO)2 + 2H2O}
%     \end{center}
%     \vspace{5pt}
%     次氯酸钙可以和盐酸反应,生成氯化钙和次氯酸:
%     \begin{center}
%         \ce{Ca(ClO)2 + 2HCl -> CaCl2 + 2HClO}
%     \end{center}
%     \vspace{5pt}
%     次氯酸钙可以和二氧化碳和水反应,生成碳酸氢钙和次氯酸:
%     \begin{center}
%         \ce{Ca(ClO)2 + 2CO2 + 2H2O -> Ca(HCO3)2 + 2HClO}
%     \end{center}
%     \vspace{5pt}
%     次氯酸具有消毒和漂白的作用,可以通过氯气和水的反应获得,
%     也可以通过次氯酸盐和酸的反应获得,但是氯气作为气体不易储存,
%     而次氯酸盐则是固体,且比较稳定,易于储存和运输。\\[3mm]
%     漂白粉实际上就是氯化钙和次氯酸钙的混合物,其中有效成分是次氯酸钙。\\[5mm]
%     实验室中,氯气通常用二氧化锰和浓盐酸的反应制备:
%     \begin{center}
%         \ce{MnO2 + 4HCl \text{(浓)} ->T[加热] MnCl2 + Cl2 ^ + 2H2O}
%     \end{center}

% \subsubsection{盐酸}
%     盐酸(HCl)是一种一元强酸,无色液体,稀盐酸和浓盐酸均为非氧化性酸。\\[3mm]
%     稀盐酸可以与活泼金属反应:
%     \begin{center}
%         \ce{Fe + 2HCl -> FeCl2 + H2 ^}\\[3mm]
%         \ce{Zn + 2HCl -> ZnCl2 + H2 ^}
%     \end{center}
%     \vspace{5pt}
%     稀盐酸可以和碱发生反应:
%     \begin{center}
%         \begin{tabular}{rl}
%             &\ce{HCl + NaOH -> NaCl + H2O}\\[3mm]
%             &\ce{HCl + KOH -> KCl + H2O}\\[3mm]
%             &\ce{HCl + Cu(OH)2 -> CuCl2 + H2O}
%         \end{tabular}
%     \end{center}
%     \vspace{5pt}
%     稀盐酸可以和碱酐发生反应:
%     \begin{center}
%         \begin{tabular}{rl}
%             &\ce{2HCl + CuO -> CuCl2 + H2O}\\[3mm]
%             &\ce{2HCl + CaO -> CaCl2 + H2O}\\[3mm]
%             &\ce{6HCl + Al2O3 -> 2AlCl3 + 3H2O}\\[3mm]
%         \end{tabular}
%     \end{center}
%     \vspace{5pt}
%     稀盐酸可以和银盐反应产生沉淀:
%     \begin{center}
%         \ce{HCl + AgNO3 -> AgCl v + HNO3}
%     \end{center}
%     \vspace{15pt}
%     盐酸实际上是氯化氢的水溶液,$1$份水大约可以溶解$500$份氯化氢。\\[3mm]
%     实验室常用浓硫酸和氯化钠的反应制备氯化氢:\vspace{3pt}
%     \begin{center}
%         \begin{tabular}{rl}
%             &\ce{H2SO4 \text{(浓)} + NaCl ->T[微热] NaHSO4 + HCl ^}\\[3mm]
%             &\ce{H2SO4 \text{(浓)} + 2NaCl ->T[强热] Na2SO4 + 2HCl ^}
%         \end{tabular}        
%     \end{center}
%     \vspace{15pt}
%     市售浓盐酸的浓度为$37\%$,密度为$1.19$~g/cm$^3$,无色液体,具有刺激性气味。\\[3mm]
%     工业盐酸由于含有铁盐等杂质,所以显黄色。

% \subsection{卤族元素}
%     以下表格列出了卤族元素的基本性质:\vspace{5pt}
%     \begin{table}[h]
%         \begin{center}
%             \begin{tabular}{l|l|l|l|l}
%                 \hline
%                 名称\quad\qquad&符号\quad\qquad&单质状态\qquad\qquad&单质颜色\qquad\qquad&水溶液颜色\qquad\\ \hline
%                 氟&\ce{F}&气体&淡黄绿色&反应\\ \hline
%                 氯&\ce{Cl}&气体&黄绿色&黄绿色\\ \hline
%                 溴&\ce{Br}&液体&红棕色&橙色\\ \hline
%                 碘&\ce{I}&固体&紫黑色&棕色\\ \hline
%             \end{tabular}
%             \caption{卤族元素的基本性质}
%         \end{center}
%     \end{table}\\
%     卤素可以和氢气发生反应:\\[3mm]
%     氟和氢气可以在阴暗处剧烈反应,发生爆炸:
%     \begin{center}
%         \ce{F2 + H2 -> 2HF + 572.2 kJ}
%     \end{center}
%     \vspace{5pt}
%     氯和氢气可以在光照下剧烈反应,发生爆炸:
%     \begin{center}
%         ~\ce{Cl2 + H2 -> 2HCl + 92.3 kJ}
%     \end{center}
%     \vspace{5pt}
%     溴和氢气需要在加热的条件下缓慢反应:
%     \begin{center}
%         ~\ce{Br2 + H2 -> 2HBr + 36.4 kJ}
%     \end{center}
%     \vspace{5pt}
%     碘和氢气需要在高温的条件下缓慢反应:
%     \begin{center}
%         \ce{I2 + H2 <=> 2HBr - 51.9 kJ}
%     \end{center}
%     \vspace{15pt}
%     卤素的活泼性按照氟氯溴碘的顺序依次递减。

% \newpage

% \section{氧~~硫}

% \subsection{硫}
%     硫,淡黄色晶体,
%     熔点$112^\circ$C,
%     沸点$444^\circ$C。\\[3mm]
%     硫是一种主族元素,硫通常显$-2$价,有时也可以显$+4$价,$+6$价。\\[5mm]
%     硫可以和铁反应,生成黑褐色的硫化亚铁:
%     \begin{center}
%         \ce{Fe + S ->T[加热] FeS}\\[3mm]
%     \end{center}
%     硫可以和铜反应,生成黑色的硫化亚铜:
%     \begin{center}
%         \ce{2Cu + S ->T[加热] Cu2S}\\[5mm]
%     \end{center}
%     硫的氧化性较弱,所以和金属反应通常只能氧化至金属的低价态,同时需要加热。\\[6mm]
%     硫可以和汞反应,生成黑色的硫化汞:
%     \begin{center}
%         \ce{Hg + S -> HgS}\\[3mm]
%     \end{center}
%     汞是其中的一个特例,在硫和汞的反应中并不需要加热,而汞也直接被氧化至高价态。\\[3mm]
%     这实际上是由于汞的物理特性造成的,汞常温下是液体,
%     液体状态的汞可以和固体状态的硫更加充分的接触,
%     使反应进行的更加完全,因此出现了这样一个特例。\\[3mm]
%     实验室常用硫粉处理意外散落的汞滴,虽然硫化汞仍然有毒,但硫化汞的毒性低于汞,且不会产生汞蒸气,
%     同时固体的硫化汞相较于液体的汞更容易收集。\\[5mm]
%     硫可以和氢气在加热的条件下反应,生成有臭鸡蛋气味的硫化氢:
%     \begin{center}
%         \ce{S + H2 <=>T[加热] H2S}\\[3mm]
%     \end{center}
%     硫可以和氧气在点燃的条件下反应,生成有刺激性气味的二氧化硫:
%     \begin{center}
%         \ce{S + O2 ->T[点燃] SO2}\\[3mm]
%     \end{center}
%     硫在少量氧气中燃烧时呈淡蓝色火焰,硫在足量氧气中燃烧时呈蓝紫色火焰。\\[5mm]
%     硫可以和亚硫酸钠反应,生成硫代硫酸钠:
%     \begin{center}
%         \ce{S + Na2SO3 ->T[加热] Na2S2O3}\\[3mm]
%     \end{center}
%     碳酸氢钠俗称小苏打,硫代硫酸钠俗称大苏打。

% \newpage

% \subsubsection{硫的氢化物}
%     硫的氢化物:硫化氢(\ce{H2S})。\\[3mm]
%     硫化氢是一种具有臭鸡蛋气味的无色气体,其中硫呈$-2$价。\\[5mm]
%     硫化氢中硫处于最低价,具有还原性。\\[3mm]
%     硫化氢可以被卤素单质氧化至$+0$价:
%     \begin{center}
%         \begin{tabular}{rl}
%             &\ce{H2S + Cl2 -> 2HCl + S v}\\[3mm]
%             &\ce{H2S + Br2 -> 2HBr + S v}\\[3mm]
%             &\ce{H2S + I2 -> 2HI + S v}\\[3mm]            
%         \end{tabular}
%     \end{center}
%     硫化氢可以被过量饱和氯水氧化至$+6$价:
%     \begin{center}
%         \ce{H2S + 4Cl2 + 4H2O -> H2SO4 + 8HCl}\\[8mm]
%     \end{center}
%     硫化氢可以被中性高锰酸钾氧化至$+0$价:
%     \begin{center}
%         \ce{2KMnO4 + 3H2S -> 2MnO2 + 2KOH + 3S + 2H2O}\\[3mm]
%     \end{center}
%     硫化氢可以被酸性高锰酸钾氧化至$+6$价:
%     \begin{center}
%         \ce{8KMnO4 + 5H2S + 7H2SO4 -> 8MnSO4 + 4K2SO4 + 12H2O}\\[8mm]
%     \end{center}
%     硫化氢是可燃性气体,燃烧时呈蓝色火焰。\\[3mm]
%     硫化氢在氧气充足时燃烧:
%     \begin{center}
%         \ce{2H2S + 3O2 ->T[点燃] 2SO2 + 2H2O}\\[3mm]
%     \end{center}
%     硫化氢在氧气不足时燃烧:
%     \begin{center}
%         \ce{2H2S + O2 ->T[点燃] 2S + 2H2O}\\[8mm]
%     \end{center}  
%     硫化氢常用醋酸铅试制检验:
%     \begin{center}
%         \ce{Pb(Ac)2 + H2S -> PbS v + 2HAc}\\[3mm]
%     \end{center}
%     由于反应产生的硫化铅呈棕黑色,所以会观察到醋酸铅试制由白变黑。\\[5mm]
%     硫化氢可以通过硫化亚铁和酸的反应制取:\vspace{3pt}
%     \begin{center}
%         \begin{tabular}{rl}
%             &\ce{FeS + 2HCl -> FeCl2 + H2S ^}\\[3mm]
%             &\ce{FeS + 2H2SO4 -> FeSO4 + H2S ^}\\[3mm]
%         \end{tabular}
%     \end{center}

% \newpage

% \subsubsection{硫的氧化物}
%     硫的氧化物有两种:二氧化硫(SO2),三氧化硫(SO3)。\\[4mm]
%     二氧化硫是一种具有刺激性气味的无色气体,其中硫呈$+4$价。\\[2mm]
%     三氧化硫是一种具有刺激性气味的无色气体,其中硫呈$+6$价。\\[6mm]
%     二氧化硫中的硫元素处于中间价,具有还原性:\\[3mm]
%     二氧化硫可以和氧气反应,生成三氧化硫:
%     \begin{center}
%         \ce{2SO2 + O2 <=>T[催化剂][加热] 2SO3}\\[3mm]
%     \end{center}
%     二氧化硫可以和氯水反应,生成硫酸和氢氯酸:
%     \begin{center}
%         \ce{2SO2 + Cl2 + 2H2O -> H2SO4 + 2HCl}\\[3mm]
%     \end{center}
%     二氧化硫可以和溴水反应,生成硫酸和氢溴酸:
%     \begin{center}
%         \ce{2SO2 + Br2 + 2H2O -> H2SO4 + 2HBr}\\[3mm]
%     \end{center}
%     二氧化硫可以和碘水反应,生成硫酸和氢碘酸:
%     \begin{center}
%         \ce{2SO2 + I2 + 2H2O -> H2SO4 + 2HI}\\[3mm]
%     \end{center}
%     上述反应中,均会观察到溶液褪色,变为无色。\\[5mm]
%     二氧化硫可以和酸性高锰酸钾反应:
%     \begin{center}
%         \ce{5SO2 + 2KMnO4 + 2H2O -> 2H2SO4 + 2MnSO4 + K2SO4}\\[3mm]
%     \end{center}
%     观察到紫红色溶液褪色,变为无色。\\[8mm]
%     二氧化硫中的硫元素处于中间价,具有氧化性:\\[3mm]
%     二氧化硫可以和硫化氢反应,生成硫和水:
%     \begin{center}
%         \ce{SO2 + 2H2S -> 3S v + 2H2}\\[3mm]
%     \end{center}
%     该试验通常使用两个集气瓶,一瓶装二氧化硫气体,一瓶装硫化氢气体,中间用玻璃板隔开,
%     实验时抽出玻璃板,使气体充分的混合,观察到集气瓶的瓶壁上有淡黄色粉末和小水滴生成。\\[10mm]
%     二氧化硫通入品红溶液,观察到溶液褪色,加热褪色的品红溶液,观察到溶液重新变为红色。\\[3mm]
%     该实验说明了二氧化硫具有漂白性,可以与部分有色物质结合形成无色物质,
%     但生成的无色物质不稳定,受热或久置后均会分解,重新变为原来的有色物质。

% \newpage

%     三氧化硫和水发生剧烈反应,生成硫酸:
%     \begin{center}
%         \ce{SO3 + H2O -> H2SO4}\\[4mm]
%     \end{center}
%     二氧化硫和水发生可逆反应,生成亚硫酸:
%     \begin{center}
%         \ce{SO2 + H2O <=> H2SO3}\\[4mm]
%     \end{center}
%     亚硫酸为中强酸,具有酸的通性:
%     \begin{center}
%         \ce{H2SO3 + 2NaOH -> Na2SO3 + 2H2O}\\[4mm]
%     \end{center}
%     亚硫酸具有还原性,可以被氧气氧化为硫酸:
%     \begin{center}
%         \ce{2H2SO3 + O2 -> 2H2SO4}\\[4mm]
%     \end{center}
%     亚硫酸具有氧化性,可以被硫化氢还原为硫:
%     \begin{center}
%         \ce{H2SO3 + 2H2S -> 3S v + 3H2O}\\[4mm]
%     \end{center}
%     实验室常用浓硫酸和亚硫酸钠的反应制备二氧化硫:
%     \begin{center}
%         \ce{Na2SO3 + H2SO4\text{(浓)} -> Na2SO4 + SO2 ^ + H2O}\\[4mm]
%     \end{center}

% \subsubsection{硫酸}
%     硫酸(\ce{H2SO4})是一种二元强酸,无色液体,稀硫酸为非氧化性酸,浓硫酸为强氧化性酸。\\[3mm]
%     稀硫酸可以和钡盐反应产生沉淀:
%     \begin{center}
%         \ce{H2SO4 + BaCl2 -> BaSO4 v + 2HCl}\\[5mm]
%     \end{center}
%     市售浓硫酸的浓度为$98\%$,密度为$1.84g/cm^3$,无色油状液体。\\[3mm]
%     浓硫酸具有吸水性,具体来说,可以将浓硫酸装入洗气瓶中,用于干燥气体。\\[3mm]
%     浓硫酸具有脱水性,具体来说,浓硫酸可以以$2:1$的比例从有机物中夺取氢和氧并形成水。\\[4mm]
%     浓硫酸和蔗糖的反应:
%     \begin{center}
%         \ce{C12H22O11 ->T[浓\ce{H2SO4}] 12C + 11H2O}\\[6mm]
%     \end{center}
%     实验中在蔗糖中加入浓硫酸后观察到,颜色逐渐变黑,体积剧烈膨胀,形成疏松多孔的炭。\\[4mm]
%     浓硫酸在稀释过程中会剧烈放热并伴有液体飞溅,所以稀释过程需要有严格的操作规范,
%     准备好装有水的烧杯,将浓硫酸缓缓倒入烧杯中,并用玻璃杯不断搅拌使得热量尽可能的扩散,
%     避免热量在一处集中造成液体飞溅。

% \newpage

%     浓硫酸可以与活泼金属锌反应,生成硫化氢气体:
%     \begin{center}
%         \ce{4Zn + 5H2SO4\text{(浓)} ->T[加热] 4ZnSO4 + H2S ^ + 4H2O}\\[4mm]
%     \end{center}
%     浓硫酸可以与不活泼金属铜反应,生成二氧化硫气体:
%     \begin{center}
%         \ce{Cu + 2H2SO4\text{(浓)} ->T[加热] CuSO4 + SO2 ^ + 2H2O}\\[4mm]
%     \end{center}
%     在该反应中,一份硫酸表现出强氧化性,一份硫酸表现出酸性。\\[3mm]
%     我们可以将反应拆分为三步理解:\\[3mm]
%     1.浓硫酸和铜发生氧化还原反应:
%     \begin{center}
%         \ce{Cu + H2SO4\text{(浓)} -> CuO + H2SO3}\\[3mm]
%     \end{center}
%     2.浓硫酸和氧化铜发生非氧化还原反应:
%     \begin{center}
%         \ce{Cu + H2SO4\text{(浓)} -> CuSO4 + H2O}\\[3mm]
%     \end{center}
%     3.生成的亚硫酸发生分解,产生水和二氧化硫:
%     \begin{center}
%         \ce{H2SO3 -> H2O + SO2 ^}\\[8mm]
%     \end{center}
%     浓硫酸可以与非金属碳反应:
%     \begin{center}
%         \ce{C + 2H2SO4\text{(浓)} ->T[加热] CO2 ^ + 2SO2 ^ + 2H2O}\\[3mm]
%     \end{center}
%     在该反应中,一份二氧化碳是由碳被浓硫酸氧化得到的,两份二氧化硫是由亚硫酸分解得到的。\\[5mm]
%     浓硫酸可以与非金属硫反应:
%     \begin{center}
%         \ce{S + 2H2SO4\text{(浓)} ->T[加热] SO2 ^ + 2SO2 ^ + 2H2O}\\[3mm]
%     \end{center}
%     在该反应中,一份二氧化硫是由硫被浓硫酸氧化得到的,两份二氧化硫是由亚硫酸分解得到的。\\[5mm]
%     我们可以将反应拆分为两步理解:\\[3mm]
%     1.碳和浓硫酸发生氧化还原反应:
%     \begin{center}
%         \ce{C + 2H2SO4\text{(浓)} -> CO2 ^ + 2H2SO3}\\[3mm]
%     \end{center}
%     2.生成的亚硫酸发生分解,产生水和二氧化硫:
%     \begin{center}
%         \ce{H2SO3 -> H2O + SO2 ^}\\[3mm]
%     \end{center}
%     1.硫和浓硫酸发生氧化还原反应:
%     \begin{center}
%         \ce{S + 2H2SO4\text{(浓)} -> SO2 ^ + 2H2SO3}\\[3mm]
%     \end{center}
%     2.生成的亚硫酸发生分解,产生水和二氧化硫:
%     \begin{center}
%         \ce{H2SO3 -> H2O + SO2 ^}
%     \end{center}    

% \newpage
%     浓硫酸和硫化氢反应时,浓硫酸过量:
%     \begin{center}
%         \ce{H2S + 3H2SO4 -> 4H2O + 4SO2 ^}\\[3mm]
%     \end{center}
%     浓硫酸和硫化氢反应时,浓硫酸适量:
%     \begin{center}
%         \ce{H2S + H2SO4 -> 2H2O + S v + SO2 ^}\\[3mm]
%     \end{center}
%     浓硫酸和硫化氢反应时,浓硫酸适量:
%     \begin{center}
%         \ce{3H2S + H2SO4 -> 4H2O + 3S v}\\[3mm]
%     \end{center}
%     浓硫酸可以与盐酸盐反应:
%     \begin{center}
%         \ce{NaCl + H2SO4\text{(浓)} ->T[微热] NaHSO4 + HCl ^}\\[3mm]
%         \ce{2NaCl + H2SO4\text{(浓)} ->T[强热] Na2SO4 + 2HCl ^}\\[3mm]
%     \end{center}
%     浓硫酸可以与硝酸盐反应:
%     \begin{center}
%         \ce{NaNO3 + H2SO4\text{(浓)} ->T[微热] NaHSO4 + HNO3 ^}\\[3mm]
%         \ce{2NaNO3 + H2SO4\text{(浓)} ->T[强热] Na2SO4 + 2HNO3 ^}\\[3mm]
%     \end{center}
%     硫酸是难挥发性酸,硝酸和盐酸是易挥发性酸,而上述两组反应均是非氧化还原反应,
%     所以两者均体现了难挥发性酸制易挥发性酸的规律。\\[5mm]
%     稀硫酸与氯盐的反应为复分解反应:
%     \begin{center}
%         \ce{2NaCl + H2SO4\text{(稀)} ->T[加热] Na2SO4 + 2HCl ^}\\[4mm]
%     \end{center}
%     浓硫酸与氯盐的反应为复分解反应:
%     \begin{center}
%         \ce{2NaCl + H2SO4\text{(浓)} ->T[加热] Na2SO4 + 2HCl ^}\\[4mm]
%     \end{center}
%     稀硫酸与溴盐和碘盐的反应为复分解反应:
%     \begin{center}
%         \ce{2NaBr + H2SO4\text{(稀)} ->T[加热] Na2SO4 + 2HBr ^}\\[4mm]
%         \ce{2NaI + H2SO4\text{(稀)} ->T[加热] Na2SO4 + 2HI ^}\\[4mm]
%     \end{center}
%     浓硫酸与溴盐和碘盐的反应为氧化还原反应:
%     \begin{center}
%         \ce{2NaBr + H2SO4\text{(浓)} ->T[加热] Na2SO4 + Br2 ^ + SO2 ^ + 2H2O}\\[4mm]
%         \ce{2NaI + H2SO4\text{(浓)} ->T[加热] Na2SO4 + I2 ^ + SO2 ^ + 2H2O}\\[4mm]
%     \end{center}
%     上述反应的实质是浓硫酸不能氧化氢氯酸,但是可以氧化氢溴酸和氢碘酸:
%     \begin{center}
%         \ce{2HBr + H2SO4\text{(浓)} ->T[加热] Br2 ^ + SO2 ^ + 2H2O}\\[4mm]
%         \ce{2HI + H2SO4\text{(浓)} ->T[加热] I2 ^ + SO2 ^ + 2H2O}\\[4mm]
%     \end{center}

% \newpage

%     工业制取硫酸分为三步:\\[3mm]
%     \textbf{1.二氧化硫的制取(沸腾炉)}\\[3mm]
%     二氧化硫的制取可以使用硫磺:
%     \begin{center}
%         \ce{S + O2 ->T[点燃] SO2}\\[3mm]
%     \end{center}
%     二氧化硫的制取可以使用黄铁矿:
%     \begin{center}
%         \ce{4FeS2 + 11O2 ->T[高温] 2Fe2O3 + 8SO2}\\[5mm]
%     \end{center}
%     所用的矿石需要粉碎,提高燃烧效率,使反应可以充分进行。\\[3mm]
%     生成的二氧化硫需要净化,避免导致后续反应中发生催化剂中毒。\\[4mm]
%     \textbf{2.二氧化硫的催化氧化(接触室)}\\[3mm]
%     二氧化硫的催化氧化:
%     \begin{center}
%         \ce{2SO2 + O2 <=>T[高温][催化剂] 2SO3}\\[3mm]
%     \end{center}
%     该反应是一个放热反应,
%     所以在实际工业生产中为了节约能量,
%     会将二氧化硫和空气的混合气体先行通过热交换器进行预热,
%     充分利用了反应时放出的废热。\\[3mm]
%     二氧化硫催化氧化的催化剂,既可以使用五氧化二钒(\ce{V2O5}),也可以使用铂(\ce{Pt})。\\[4mm]
%     \textbf{3.三氧化硫的吸收(合成塔)}\\[3mm]
%     三氧化硫的吸收:
%     \begin{center}
%         \ce{SO3 + H2O -> H2SO4}\\[3mm]
%     \end{center}
%     工业生产时,并不会使用水吸收,这是因为两者会剧烈反应,
%     形成大量细小的硫酸珠滴,即酸雾,
%     而当排出废气时,这部分酸雾也会一并排出,
%     造成硫酸的大量损失。\\[3mm]
%     工业生产时,会用浓硫酸吸收,使得三氧化硫直接溶于浓硫酸,
%     形成发烟硫酸,获得发烟硫酸后稀释就可以得到任意浓度的硫酸。\\[6mm]
%     硫酸盐构成的晶体称为矾,以下是几种常见的硫酸盐晶体:\vspace{5pt}
%     \begin{table}[h]
%         \begin{center}
%             \begin{tabular}{l|l|l}
%                 \hline
%                 胆矾\qquad\qquad&\ce{CuSO4 * 5H2O}\qquad\qquad&蓝色晶体\qquad\qquad\\ \hline
%                 绿矾\qquad\qquad&\ce{FeSO4 * 7H2O}\qquad\qquad&绿色晶体\qquad\qquad\\ \hline
%                 皓矾\qquad\qquad&\ce{ZnSO4 * 7H2O}\qquad\qquad&无色晶体\qquad\qquad\\ \hline
%                 明矾\qquad\qquad&\ce{KAl(SO4)2 * 12H2O}\qquad\qquad&无色晶体\qquad\qquad\\ \hline
%             \end{tabular}
%         \caption{硫酸盐晶体}
%         \end{center}
%     \end{table}    

% \newpage

% \section{氮~~磷}

% \subsection{氮}
%     氮,无色气体,熔点$-210^\circ$C,沸点$-196^\circ$C。\\[3mm]
%     氮是一种主族元素,氮通常显$-3$价,有时也可以显$+1$价,$+2$价,$+3$价,$+4$价,$+5$价。\\[5mm]
%     氮气中存在三根共价键,所以化学性质较为稳定,通常不与任何物质反应,
%     但在某些条件下,氮分子获取了足够的能量,也能与部分物质发生反应。\\[3mm]
%     氮气与氢气的反应:
%     \begin{center}
%         \ce{N2 + 3H2 <=>T[高温~~高压][催化剂] 2NH3}\\[4mm]
%     \end{center}
%     氮气与氧气的反应:
%     \begin{center}
%         \ce{N2 + O2 ->T[电火花] 2NO}\\[4mm]
%     \end{center}

% \subsubsection{氮的氢化物}
%     氮的氢化物:氨(\ce{NH3})。\\[3mm]
%     氨气是一种具有刺激性气味的有毒气体,其中氮呈$-3$价。\\[5mm]
%     氨气与水的反应生成一水合氨:
%     \begin{center}
%         \ce{NH3 + H2O <=> NH3 * H2O}\\[4mm]
%     \end{center}
%     一水合氨受热分解产生氨气和水:
%     \begin{center}
%         \ce{NH3 * H2O ->T[加热] NH3 ^ + H2O}\\[4mm]
%     \end{center}
%     我们将氨气的水溶液称为氨水,当氨气溶于水时,有部分氨气会与水发生反应产生一水合氨,
%     而一水合氨可以部分电离出氢氧根离子,故氨水呈碱性。\\[5mm]
%     氨气可以与酸反应,生成对应的铵盐:
%     \begin{center}
%         \ce{NH3 + HCl -> NH4Cl}\\[3mm]
%         \ce{NH3 + HNO3 -> NH4NO3}\\[3mm]
%     \end{center}
%     如果我们使用两根玻璃棒分别蘸取浓氨水和浓盐酸或浓硝酸,然后使两根玻璃杯相互接近但不接触,
%     会观察到玻璃棒之间产生白烟,这是由于浓氨水和浓盐酸或浓硝酸均具有强烈的挥发性,
%     两者的挥发物直接在空中反应产生了对应的铵盐,故观察到了白烟的现象。
    
% \newpage

%     氨气可以被氯气氧化至$0$价,生成氮气:
%     \begin{center}
%         \ce{3Cl2 + 8NH3 -> N2 + 6NH4Cl}\\[4mm]
%     \end{center}
%     氨气可以被溴气氧化至$0$价,生成氮气:
%     \begin{center}
%         \ce{3Br2 + 8NH3 -> N2 + 6NH4Br}\\[4mm]
%     \end{center}
%     该反应伴有发烟现象,常用于寻找氯气管道的漏气处。\\[4mm]
%     氨气可以被氧气氧化至$+2$价,生成一氧化氮:
%     \begin{center}
%         \ce{5O2 + 4NH3 ->T[催化剂][加热] 4NO + 6H2O}\\[4mm]
%     \end{center}
%     该反应也被称为氨气的催化氧化,常用于工业制取硝酸。\\[10mm]
%     氯化氨受热易分解,产生氨气和盐酸:
%     \begin{center}
%         \ce{NH4Cl ->T[加热] NH3 ^ + HCl ^}\\[4mm]
%     \end{center}
%     碳酸氨受热易分解,产生氨气和碳酸:
%     \begin{center}
%         \ce{(NH4)2CO3 ->T[加热] 2NH3 ^ + CO2 ^ + H2O}\\[4mm]
%     \end{center}
%     硝酸铵在加热或猛烈撞击时会发生爆炸性分解,因此可以用作炸药:
%     \begin{center}
%         \ce{2NH4NO3 ->T[加热] 2N2 ^ + O2 ^ + 4H2O}\\[10mm]
%     \end{center}
%     氯化铵可以和氢氧化钙反应,产生氨气:
%     \begin{center}
%         \ce{Ca(OH)2 + 2NH4Cl ->T[加热] CaCl2 + 2NH3 ^ + 2H2O}\\[4mm]
%     \end{center}
%     硫酸铵可以和氢氧化钙反应,产生氨气:
%     \begin{center}
%         \ce{Ca(OH)2 + (NH4)2SO4 ->T[加热] CaSO4 + 2NH3 ^ + 2H2O}\\[6mm]
%     \end{center}
%     实验室常用铵盐和碱的反应制备氨气。

% \newpage

% \subsubsection{氮的氧化物}
%     氮的氧化物有六种:一氧化氮(\ce{NO}),二氧化氮(\ce{NO2}),一氧化二氮(\ce{N2O}),三氧化二氮(\ce{N2O3}),\\[1mm]
%     四氧化二氮(\ce{N2O4}),五氧化二氮(\ce{N2O5})。\\[6mm]
%     一氧化氮中氮呈$+2$价,是一种无色气体。\\[3mm]
%     二氧化氮中氮呈$+4$价,是一种红棕色气体。\\[6mm]
%     一氧化二氮中氮呈$+1$价,是一种无色气体。\\[3mm]
%     三氧化二氮中氮呈$+3$价,是一种红棕色气体。\\[3mm]
%     四氧化二氮中氮呈$+4$价,是一种无色气体。\\[3mm]
%     五氧化二氮中氮呈$+5$价,是一种白色固体。\\[6mm]
%     一氧化二氮和水反应,产生硝酸和氮气:
%     \begin{center}
%         \ce{5N2O + H2O -> 2HNO3 + 4N2}\\[4mm]
%     \end{center}
%     二氧化氮和水反应,产生硝酸和一氧化氮:
%     \begin{center}
%         \ce{3N2O + H2O -> 2HNO3 + NO}\\[4mm]
%     \end{center}
%     三氧化二氮和水反应,产生亚硝酸:
%     \begin{center}
%         \ce{N2O3 + H2O -> 2HNO2}\\[4mm]
%     \end{center}
%     五氧化二氮和水反应,产生硝酸:
%     \begin{center}
%         \ce{N2O5 + H2O -> 2HNO3}\\[4mm]
%     \end{center}
%     由此可见,硝酸的酸酐是五氧化二氮,亚硝酸的酸酐是三氧化二氮。\\[10mm]
%     四氧化二氮实际上是二氧化氮的二聚体形式,存在以下平衡:
%     \begin{center}
%         \ce{2NO2 <=> N2O4}\\[4mm]
%     \end{center}
%     在低温或高压的条件下,平衡向正向移动,气体的颜色偏向于无色。\\[3mm]
%     在高温或低压的条件下,平衡向逆向移动,气体的颜色偏向于红棕色。

% \newpage

%     闪电会导致大气中的氮气和氧气反应,产生一氧化氮:
%     \begin{center}
%         \ce{N2 + O2 ->T[放电] 2NO}\\[4mm]
%     \end{center}
%     闪电产生的一氧化氮极易被氧气氧化,产生二氧化氮:
%     \begin{center}
%         \ce{2NO + O2 -> 2NO2}\\[4mm]
%     \end{center}
%     雨水会与闪电所产生的二氧化氮发生反应,产生硝酸:
%     \begin{center}
%         \ce{3NO2 + H2O -> 2HNO3 + NO}\\[4mm]
%     \end{center}
%     以上就是大气固氮的过程,该过程产生的硝酸是自然界中植物生长所需的重要肥料来源。\\

% \subsubsection{硝酸}
%     硝酸(\ce{HNO3})是一种一元强酸,无色液体,稀硝酸和浓硝酸均为强氧化性酸。\\[3mm]
%     硝酸不稳定,受热或见光均会分解:
%     \begin{center}
%         \ce{4HNO3 ->T[加热] 2H2O + 4NO2 ^ + O2 ^}\\[3mm]
%         \ce{4HNO3 ->T[光照] 2H2O + 4NO2 ^ + O2 ^}\\[4mm]
%     \end{center}
%     稀硝酸可以和不活泼金属铜反应,产生一氧化氮:
%     \begin{center}
%         \ce{3Cu + 8HNO3\text{(稀)} ->T[加热] 3Cu(NO3)2 + 2NO ^ + 4H2O}\\[4mm]
%     \end{center}
%     浓硝酸可以和不活泼金属铜反应,产生二氧化氮:
%     \begin{center}
%         \ce{Cu + 4HNO3\text{(浓)} ->T[加热] Cu(NO3)2+ 2NO2 ^ + 2H2O}\\[4mm]
%     \end{center}
%     浓硝酸可以与非金属碳反应:
%     \begin{center}
%         \ce{C + 4HNO3\text{(浓)} ->T[加热] CO2 ^ + 4NO2 ^ + 2H2O}\\[4mm]
%     \end{center}
%     浓硝酸可以与非金属硫反应:
%     \begin{center}
%         \ce{S + 6HNO3\text{(浓)} ->T[加热] H2SO4 + 6NO2 ^ + 2H2O}\\[6mm]
%     \end{center}
%     稀硝酸和硫化氢反应,产生一氧化氮和硫:
%     \begin{center}
%         \ce{2HNO3\text{(稀)} + 3H2S -> 2NO + 4H2O + 3S v}\\[4mm]
%     \end{center}
%     浓硝酸和硫化氢反应,产生二氧化氮和硫酸:
%     \begin{center}
%         \ce{8HNO3\text{(浓)} + H2S -> 8NO2 + 4H2O + H2SO4}
%     \end{center}

% \newpage

%     稀硝酸和二氧化硫反应,产生一氧化氮和硫酸:
%     \begin{center}
%         \ce{2HNO3\text{(稀)} + 3SO2 + 2H2O -> 2NO + 3H2SO4}\\[4mm]
%     \end{center}
%     浓硝酸和二氧化硫反应,产生二氧化氮和硫酸:
%     \begin{center}
%         \ce{2HNO3\text{(浓)} + SO2 -> 2NO2 + H2SO4}\\[4mm]
%     \end{center}
%     硝酸钾受热易分解放出氧气,高温下可以用作氧化剂:
%     \begin{center}
%         \ce{2KNO3 ->T[加热] 2KNO2 + O2 ^}\\[4mm]
%     \end{center}
%     硝酸钾与硫粉和碳粉可以混合制成黑火药,爆炸时发生如下反应:
%     \begin{center}
%         \ce{2KNO3 + 3C + S ->T[加热] K2S + N2^ + 3CO2}\\[4mm]
%     \end{center}
%     工业制取硝酸分为四步:\\[3mm]
%     \textbf{1.氨气的合成(工业固氮)}\\[2mm]
%     工业上氨气的合成:
%     \begin{center}
%         \ce{N2 + 3H2 <=>T[高温高压][催化剂] 2NH3}\\[3mm]
%     \end{center}
%     氮气和氢气在高温高压和催化剂的条件下,反应生成氨气。\\[1mm]
%     该反应通常选用含铁的物质作为催化剂。\\[5mm]
%     \textbf{2.氨气的催化氧化(氧化塔)}\\[2mm]
%     氨气的催化氧化:
%     \begin{center}
%         \ce{4NH3 + 5O2 <=>T[高温][催化剂] 4NO + 6H2O}\\[3mm]
%     \end{center}
%     氨气和氧气在高温和催化剂的条件下,在氧化塔中反应生成一氧化氮和水蒸气。\\[1mm]
%     该反应通常选用铂铑合金作为催化剂。\\[5mm]
%     \textbf{3.一氧化氮的氧化(冷却塔)}\\[2mm]
%     一氧化氮的氧化:
%     \begin{center}
%         \ce{2NO + O2 -> 2NO2}\\[3mm]
%     \end{center}
%     一氧化氮和空气混合进入冷却塔,在冷却塔中反应生成二氧化氮。\\[5mm]
%     \textbf{4.二氧化氮的吸收(吸收塔)}\\[2mm]
%     二氧化氮的吸收:
%     \begin{center}
%         \ce{3NO2 + H2O -> 2HNO3 + NO}\\[3mm]
%     \end{center}
%     二氧化氮进入吸收塔,在吸收塔中与水反应生成硝酸,
%     同时产生的一氧化氮再被另外补充的空气氧化为二氧化氮,
%     氧化和吸收的过程如此循环,最终可以使二氧化氮被较为完全的吸收。



% \newpage

% \section{铁~~铜}

% \subsection{铁}
%     铁,银白色金属,
%     熔点1535$^{\circ}$C,
%     沸点2750$^{\circ}$C。\\[3mm]
%     铁是一种副族元素,有多种化合价,
%     铁通常显$+2$价或$+3$价,铁处于$+3$价时较为稳定。\\[3mm]
%     铁和弱氧化剂反应时会变为$+2$价:
%     \begin{center}
%         \ce{Fe + S ->T[加热] FeS}\\[1mm]
%         \ce{Fe + I2 ->T[加热] FeI2}
%     \end{center}
%     \vspace{5pt}
%     铁和强氧化剂反应时会变为$+3$价:
%     \begin{center}
%         \ce{2Fe + 3Br2 ->T[加热] 2FeBr3}\\[1mm]
%         \ce{2Fe + 3Cl2 ->T[加热] 2FeCl3}
%     \end{center}
%     \vspace{20pt}
%     铁和氧气可以在点燃的条件下反应,生成四氧化三铁:
%     \begin{center}
%         \ce{3Fe + 2O2 ->T[点燃] Fe3O4}
%     \end{center}
%     \vspace{10pt}
%     铁和水可以在高温的条件下反应,生成四氧化三铁和氢气:
%     \begin{center}
%         \ce{3Fe + H2O ->T[高温] Fe3O4 + 4H2}
%     \end{center}
%     \vspace{15pt}

% \subsubsection{铁的氧化物}
%     铁的氧化物有三种:氧化亚铁(\ce{FeO}),
%     氧化铁(\ce{Fe2O3}),四氧化三铁(\ce{Fe3O4})。\\[4mm]
%     氧化亚铁是一种黑色粉末,其中铁呈$+2$价。\\[2mm]
%     氧化铁是一种红棕色粉末,其中铁呈$+3$价。\\[2mm]
%     四氧化三铁是一种黑色晶体,具有磁性,
%     其中一份铁呈$+2$价,两份铁呈$+3$价。\\[4mm]
%     氧化亚铁和氧化铁均可以和酸反应,生成亚铁盐和铁盐:
%     \begin{center}
%         \ce{FeO + H2SO4 -> FeSO4 + H2O}\\[1mm]
%         \ce{Fe2O3 + 3H2SO4 -> Fe2(SO4)3 + 3H2O}\\[5mm]
%     \end{center}
%     氧化亚铁不稳定,在空气中加热会迅速被氧化为四氧化三铁。

% \newpage

% \subsubsection{铁的氢氧化物}
%     铁的氢氧化物有两种:氢氧化亚铁(\ce{Fe(OH)2}),氢氧化铁(\ce{Fe(OH)3})。\\[3mm]
%     铁的氢氧化物均可以通过其可溶性盐和强碱反应制取。\\[3mm]
%     在氯化铁溶液中滴入氢氧化钠,观察到产生了红褐色的氢氧化铁沉淀:
%     \begin{center}
%         \ce{FeCl3 + 3NaOH -> Fe(OH)3 v + 3NaCl}\\[3mm]
%     \end{center}    
%     在硫酸亚铁溶液中滴入氢氧化钠,观察到产生了白色的氢氧化亚铁沉淀:
%     \begin{center}
%         \ce{FeSO4 + 2NaOH -> Fe(OH)2 v + 2Na2SO4}\\[4mm]
%     \end{center}    
%     需要注意的是,氢氧化亚铁极易被氧化为氢氧化铁,
%     如果在制备氢氧化亚铁的反应中没有做特别的措施避免其被氧化,
%     我们会观察到白色沉淀迅速变为灰绿色,
%     最终缓慢的变为红褐色。
%     \begin{center}
%         \ce{4Fe(OH)2 + O2 + 2H2O -> 4Fe(OH)3}\\[3mm]
%     \end{center}
%     其中所观察到的灰绿色沉淀是一种介于氢氧化亚铁和氢氧化铁间的特殊物质。\\[10mm]
%     加热氢氧化铁,氢氧化铁就会失去水分,得到它的碱酐氧化铁:
%     \begin{center}
%         \ce{2Fe(OH)3 ->T[加热] 2Fe2O3 + 3H2O}\\[3mm]
%     \end{center}
%     加热氢氧化亚铁,氢氧化亚铁就会失去水分,得到它的碱酐氧化亚铁:
%     \begin{center}
%         \ce{Fe(OH)2 ->T[加热] FeO + H2O}\\[3mm]
%     \end{center}
%     需要注意的是,由于氢氧化亚铁极易被氧化,
%     所以加热时必须隔绝空气才可以得到氧化亚铁,
%     否则会直接得到氧化铁。\\[10mm]
%     氢氧化亚铁和氢氧化铁均是难溶性碱,
%     它们能与酸反应,得到亚铁盐或铁盐:
%     \begin{center}
%         \ce{Fe(OH)2 + 2HCl -> FeCl2 + 2H2O}\\[3mm]
%         \ce{Fe(OH)3 + 3HCl -> FeCl3 + 3H2O}
%     \end{center}

% \newpage

% \subsubsection{铁离子与亚铁离子}
%     铁离子为$+3$价,在水溶液中为棕黄色。\\[3mm]
%     亚铁离子为$+2$价,在水溶液中为浅绿色。\\[5mm]
%     铁与非氧化性酸反应得到$+2$价的亚铁离子:
%     \begin{center}
%         \ce{Fe + 2HCl -> FeCl2 + H2 ^}\\[4mm]
%     \end{center}
%     关于铁和铜以及其离子的氧化性强弱有以下规律:\\[1mm]
%     氧化性:\ce{Fe^{3+}}~~>~~\ce{Cu^{2+}}~~>~~\ce{Fe^{2+}}~~>~~\ce{Cu}~~>~~\ce{Fe}\\[4mm]
%     铁和铜离子发生反应:
%     \begin{center}
%         \ce{Fe + CuCl2 -> FeCl2 +Cu}\\[3mm]
%     \end{center}
%     铁和铁离子发生反应:
%     \begin{center}
%         \ce{Fe + 2FeCl3 -> FeCl2 + 2FeCl2}\\[3mm]
%     \end{center}
%     铜和铁离子发生反应:
%     \begin{center}
%         \ce{Cu + 2FeCl3 -> CuCl2 + 2FeCl2}\\[8mm]
%     \end{center}
%     关于铁和卤素以及其离子的氧化性强弱有以下规律:\\[1mm]
%     氧化性:\ce{Cl2}~~>~~\ce{Br2}~~>~~\ce{Fe^{3+}}~~>~~\ce{I2}~~>~~\ce{S2}~~>~~\ce{Cl-}~~>~~\ce{Br-}~~>~~\ce{Fe^{2+}}~~>~~\ce{I-}~~>~~\ce{S^{2-}}~~>~~\ce{Fe}\\[4mm]
%     铁和卤素单质反应:
%     \begin{center}
%         \ce{Fe + S ->T[加热] FeS}\\[2mm]
%         \ce{Fe + I2 ->T[加热] FeI2}\\[2mm]
%         \ce{2Fe + 3Br2 ->T[加热] 2FeBr3}\\[2mm]
%         \ce{2Fe + 3Cl2 ->T[加热] 2FeCl3}\\[6mm]
%     \end{center}
%     亚铁离子和强氧化剂反应得到铁离子:
%     \begin{center}
%         \ce{2FeI2 + 3Cl2 -> 2FeCl3 + 2I2}\\[2mm]
%         \ce{2FeI2 + 3Br2 -> 2FeBr3 + 2I2}\\[6mm]
%     \end{center}
%     铁离子和强还原剂反应得到亚铁离子:
%     \begin{center}
%         \ce{2FeCl3 + H2S -> 2FeCl2 + 2HCl + S}
%     \end{center}

% \newpage

% \subsection{铜}
%     铜,紫红色金属,熔点1083$^\circ$C,沸点2595$^\circ$C。\\[3mm]
%     铜是一种副族元素,有多种化合价,铜通常显$+1$价或$+2$价,铜处于$+2$价时较为稳定。\\[5mm]
%     铜在高温下可以于氧气反应,生成黑色的氧化铜:
%     \begin{center}
%         \ce{2Cu + O2 ->T[加热] 2CuO}
%     \end{center}
%     铜在常温下可以与潮湿的空气发生反应,生成绿色的碱式碳酸铜:
%     \begin{center}
%         \ce{2Cu + O2 + H2O + CO2 -> Cu2(OH)2CO3}
%     \end{center}
%     碱式碳酸铜也被称为铜绿。\\[3mm]
%     铜具有良好的耐腐蚀性,无法与非氧化性酸反应,但是可以于强氧化性酸反应:
%     \begin{center}
%         \begin{tabular}{rl}
%             &\ce{Cu + 2H2SO4\text{(浓)} ->T[加热] CuSO4 + SO2 ^ + 2H2O}\\[3mm]
%             &\ce{Cu + 4HNO3\text{(浓)} -> Cu(NO3)2 + 2NO2 ^ + 2H2O}
%         \end{tabular}
%     \end{center}
%     \vspace{10pt}
%     铜还可以制成各种合金:
%     \begin{table}[h]
%         \begin{center}
%             \begin{tabular}{l|l}
%                 \hline
%                 青铜\qquad\qquad&铜锡合金(Cu-Sn)\qquad\qquad\\ \hline
%                 白铜\qquad\qquad&铜镍合金(Cu-Ni)\qquad\qquad\\ \hline
%                 黄铜\qquad\qquad&铜锌合金(Cu-Zn)\qquad\qquad\\ \hline
%             \end{tabular}
%             \caption{含铜的合金}
%         \end{center}
%     \end{table}\vspace{-20pt}

% \subsubsection{铜的氧化物}
%     铜的氧化物有两种:氧化亚铜(\ce{Cu2O}),氧化铜(\ce{CuO})。\\[4mm]
%     氧化亚铜是一种鲜红色粉末,其中铜呈$+1$价。\\[2mm]
%     氧化铜是一种黑色粉末,其中铜呈$+2$价。\\[4mm]
%     铜在氧气中加热,若氧气充足生成氧化铜,若氧气不足生成氧化亚铜:
%     \begin{center}
%         \ce{2Cu + O2 ->T[加热] 2CuO}\\[3mm]
%         \ce{Cu + O2 ->T[加热] Cu2O}
%     \end{center}
%     高温下,氧化铜分解为氧化亚铜和氧气:
%     \begin{center}
%         \ce{4CuO ->T[高温] 2Cu2O + O2 ^}
%     \end{center}

% \subsubsection{铜离子和亚铜离子}
%     铜离子为$+2$价,在水溶液中为蓝色。\\[3mm]
%     亚铜离子为$+1$价,在水溶液中无法稳定存在,会发生自我歧化,产生铜离子和铜。

% \newpage

% \section{镁~~铝}

% \subsection{铝}
%     铝,银白色金属,熔点660$^\circ$C,沸点2467$^\circ$C。\\[3mm]
%     铝是一种主族元素,铝通常显$+3$价。\\[5mm]
%     铝可以在氧气中被点燃,发生剧烈燃烧,产生耀眼的白光:
%     \begin{center}
%         \ce{4Al + 3O2 ->T[点燃] 2Al2O3}\\[8mm]
%     \end{center}
%     铝是一种两性金属,所以既可以与酸反应,也可以与碱反应。\\[5mm]
%     铝可以和酸发生反应:
%     \begin{center}
%         \begin{tabular}{rl}
%             &\ce{2Al + 6HCl -> 2AlCl3 + 3H2 ^}\\[3mm]
%             &\ce{2Al + 3H2SO4 -> Al2(SO4)3 + 3H2 ^}\\[3mm]
%         \end{tabular}
%     \end{center}
%     铝可以和碱发生反应:
%     \begin{center}
%         \ce{2Al + 2NaOH + 2H2O -> 2NaAlO2 + 3H2 ^}\\[4mm]
%     \end{center}
%     铝和氢氧化钠的反应实际上可以这样理解:\\[2mm]
%     铝首先与水发生可逆反应,产生氢氧化铝和氢气,
%     但这个反应实际非常微弱,同时生成的氢氧化铝不溶于水,
%     附着在铝的表面,阻止反应进一步进行。
%     \begin{center}
%         \ce{2Al + 6H2O -> 2Al(OH)3 v + 3H2 ^}\\[1mm]
%     \end{center}
%     然而随着氢氧化钠的加入,
%     氢氧化铝作为一种两性氢氧化物可以与强碱氢氧化钠反应,
%     生成偏铝酸钠和水。
%     同时由于氢氧化铝被溶解,不再有物质附着在铝表面,
%     阻止铝和水的反应进行,
%     最终使得反应可以持续进行。
%     \begin{center}
%         \ce{2Al(OH)3 + 2NaOH -> 2NaAlO2 + 4H2O}\\[8mm]
%     \end{center}
%     铝和氧化铁在高温下可以剧烈的发生反应,放出大量的热:
%     \begin{center}
%         \ce{2Al + Fe2O3 ->T[高温] 2Fe + Al2O3}
%     \end{center}
%     铝和氧化铁的反应被称为铝热反应,该反应会剧烈放热,
%     产生的热量甚至足矣维持反应自身的进行,
%     铝热反应时的温度可以高达3500$^\circ$C,
%     如此之高的温度可以使铁熔化为液态,
%     所以这个反应也常用于钢轨的焊接。\\[3mm]
%     实验室中通常会利用镁带燃烧的热量引燃铝热剂,
%     为了使镁带燃烧产生更高的温度,
%     通常还会在铝热剂中混入氯酸钾,
%     利用其高温下分解的氧气使镁带的燃烧更为剧烈。\\[3mm]
%     工业上也常用铝热法产生的高温制取难熔的金属,包括镍(\ce{Ni}),铬(\ce{Cr}),锰(\ce{Mn}),钒(\ce{V})等。

% \newpage

% \subsubsection{铝的氧化物}
%     铝的氧化物是\ce{Al2O3},氧化铝呈白色,熔沸点高,是一种较好的耐火材料。\\[3mm]
%     铝可以和氧气反应得到氧化铝:
%     \begin{center}
%         \ce{4Al + 3O2 ->T[点燃] 2Al2O3}\\[3mm]
%     \end{center}
%     刚玉是天然无色的氧化铝晶体,硬度很大,
%     若刚玉中含有铬的氧化物,则为红宝石,
%     若刚玉中含有铁或钛的氧化物,则为蓝宝石。\\[3mm]
%     电解熔融状态的氧化铝可以制得铝单质,
%     反应需要加入冰晶石(\ce{Na3AlF6})为助溶剂,
%     电解氧化铝也是工业上生产金属铝的方式。
%     \begin{center}
%         \ce{2Al2O3 ->T[通电][\ce{Na3AlF6}] 4Al + 3O2 ^}\\[8mm]
%     \end{center}
%     氧化铝是两性氧化物,既可以与酸反应,也可以与碱反应,但不与弱酸弱碱反应。\\[3mm]
%     氧化铝和酸的反应:
%     \begin{center}
%         \ce{Al2O3 + 6HCl -> 2AlCl3 + 3H2O}\\[3mm]
%     \end{center}
%     氧化铝和碱的反应:
%     \begin{center}
%         \ce{Al2O3 + 2NaOH -> 2NaAlO2 + H2O}\\[1mm]
%     \end{center}

% \subsubsection{铝的氢氧化物}
%     氢氧化铝是一种两性氢氧化物,即可以与强酸反应,也可以与强碱反应,但不与弱酸弱碱反应。\\[3mm]
%     氢氧化铝和酸的反应:
%     \begin{center}
%         \ce{Al(OH)3 + 3HCl -> AlCl3 + 3H2O}\\[3mm]
%     \end{center}
%     氢氧化铝和碱的反应:
%     \begin{center}
%         \ce{Al(OH)3 + NaOH -> NaAlO2 + 2H2O}\\[3mm]
%     \end{center}

% \subsubsection{铝盐和偏铝酸盐}
%     铝盐和弱碱反应,得到氢氧化铝:
%     \begin{center}
%         \ce{AlCl3 + 3NH4OH -> Al(OH)3 v + 3NH4Cl}\\[3mm]
%     \end{center}
%     铝盐和少量强碱反应,得到氢氧化铝:
%     \begin{center}
%         \ce{AlCl3 + 3NaOH -> Al(OH)3 v + 3NaCl}\\[3mm]
%     \end{center}
%     铝盐和过量强碱反应,得到偏铝酸盐:
%     \begin{center}
%         \ce{AlCl3 + 4NaOH -> NaAlO2 + 3NaCl + 2H2O}\\[3mm]
%     \end{center}
%     这是由于额外的一份氢氧化钠与氢氧化铝直接反应,因此产生的是偏铝酸钠。

% \newpage

%     偏铝酸盐和弱酸反应,得到氢氧化铝:
%     \begin{center}
%         \ce{NaAlO2 + HAc + H2O -> Al(OH)3 v + NaAc}\\[3mm]
%     \end{center}
%     偏铝酸盐和少量强酸反应,得到氢氧化铝:
%     \begin{center}
%         \ce{NaAlO2 + HCl + H2O -> Al(OH)3 v + NaCl}\\[3mm]
%     \end{center}
%     偏铝酸盐和过量强酸反应,得到铝盐:
%     \begin{center}
%         \ce{NaAlO2 + 4HCl -> AlCl3 + NaCl + 2H2O}\\[3mm]
%     \end{center}
%     这是由于额外的三份硫酸与氢氧化铝直接反应,因此产生的是铝盐。\\

% \subsection{镁}
%     镁,银白色金属,熔点649$^\circ$C,沸点1090$^\circ$C。\\[3mm]
%     镁是一种主族元素,镁通常显$+2$价。\\[5mm]
%     镁可以在氧气中被点燃,发生剧烈燃烧,产生耀眼的白光。
%     \begin{center}
%         \ce{2Mg + O2 -> 2MgO}
%     \end{center}
%     化学上常利用镁带燃烧产生的强光和强热进行实验。\\[6mm]
%     镁可以和酸发生反应:
%     \begin{center}
%         \ce{Mg + 2HCl -> MgCl2 + H2 ^}
%     \end{center}
%     \vspace{15pt}
%     镁可以和一些非金属氧化物发生反应:
%     \begin{center}
%         \ce{2Mg + CO2 ->T[点燃] 2MgO + C}\\[3mm]
%         \ce{2Mg + SO2 ->T[点燃] 2MgO + S}\\[3mm]
%     \end{center}
%     \vspace{15pt}
%     工业上常用煅烧碳酸镁(\ce{MgCO3})的方式制取氧化镁:
%     \begin{center}
%         \ce{MgCO3 ->T[高温] MgO + CO2 ^}\\[3mm]
%     \end{center}
%     工业上常用电解氯化镁(\ce{MgCl2})的方式制取金属镁:
%     \begin{center}
%         \ce{MgCl2 ->T[通电] Mg + Cl2 ^}\\[3mm]
%     \end{center}

% \newpage

% \section{电离和水解}
    
% \subsection{电离}
%     我们将在水中完全电离的化合物称为强电解质。\\[3mm]
%     我们将在水中部分电离的化合物称为弱电解质。\\[3mm]
%     提高温度,稀释溶液,均可以促进电离。

% \subsubsection{水的电离}
%     水是一种极弱的电解质,在纯水中会有极少的一部分水分子发生电离。
%     \begin{center}
%         \ce{H2O <=> 2H+ + OH-}
%     \end{center}

% \subsubsection{强酸的电离}
%     强酸可以在水中完全电离,属于强电解质。
%     \begin{center}
%         \ce{HCl -> H+ + Cl-}\\[3mm]
%         \ce{HNO3 -> H+ + NO3^{2-}}\\[3mm]
%         \ce{H2SO4 -> 2H+ + SO4^{2-}}
%     \end{center}

% \subsubsection{弱酸的电离}
%     弱酸可以在水中部分电离,属于弱电解质。\\[3mm]
%     弱酸的电离是分步进行的,且一步比一步困难。\\[4mm]
%     一元弱酸的电离:
%     \begin{center}
%         \ce{HF <=> H+ + F-}
%     \end{center}
%     \vspace{5pt}
%     二元弱酸的电离:
%     \begin{center}
%         \begin{tabular}{rl}
%             &\ce{H2CO3 <=> H+ + HCO3-}\\[3mm]
%             &\ce{HCO3- <=> H+ + CO3^{2-}}
%         \end{tabular}
%     \end{center}
%     \vspace{5pt}
%     三元弱酸的电离:
%     \begin{center}
%         \begin{tabular}{rl}
%             &\ce{H3PO4 <=> H+ + H2PO4-}\\[3mm]
%             &\ce{H2PO4- <=> H+ + HPO4^{2-}}\\[3mm]
%             &\ce{HPO4^{2-} <=> H+ + PO4^{3-}}\\[3mm]
%         \end{tabular}
%     \end{center}

% \newpage

% \subsubsection{强碱的电离}
%     强碱可以在水中完全电离,属于强电解质。
%     \begin{center}
%         \ce{NaOH -> Na^{+} + OH-}\\[3mm]
%         \ce{Ba(OH)2 -> Ba^{2+} + 2OH-}\\[3mm]
%         \ce{Ca(OH)2 -> Ca^{2+} + 2OH-}
%     \end{center}

% \subsubsection{弱碱的电离}
%     弱碱可以在水中部分电离,属于弱电解质。\\[3mm]
%     弱碱的电离实际上也是分步进行的,但由于较为复杂,高中阶段暂且认为弱碱是一步电离的。\vspace{3pt}
%     \begin{center}
%         \ce{Cu(OH)2 <=> Cu^{2+} + 2OH-}\\[3mm]
%         \ce{Fe(OH)3 <=> Fe^{3+} + 3OH-}\\[3mm]
%         \ce{NH3\cdot H2O <=> NH4+ + OH-}
%     \end{center}

% \subsubsection{正盐的电离}
%     正盐可以在水中完全电离,属于强电解质。
%     \begin{center}
%         \ce{NaCl -> Na+ + Cl-}\\[3mm]
%         \ce{Na2SO4 -> 2Na^{+} + SO4^{2-}}\\[3mm]
%         \ce{Na2CO3 -> 2Na^{+} + CO3^{2-}}
%     \end{center}

% \subsubsection{酸式盐的电离}
%     酸式盐可以在水中完全电离,属于强电解质。\\[3mm]
%     强酸对应的酸式盐在水中一步电离,且为完全电离。
%     \begin{center}
%         \ce{NaHSO4 -> Na^{+} + H+ + SO4^{2-}}\\[6mm]
%     \end{center}
%     弱酸对应的酸式盐在水中分步电离,第一步为完全电离,
%     产生的酸式酸根分布部分电离。\vspace{8pt}
%     \begin{center}
%         \begin{tabular}{rl}
%             &\ce{NaHCO3 -> Na^{+} + HCO3^{-}}\\[3mm]
%             &\ce{HCO3- <=> H^{+} + CO3^{2-}}\\[6mm]
%             &\ce{NaH2PO4 -> Na^{+} + H2PO4^{-}}\\[3mm]
%             &\ce{H2PO4- <=> H+ + HPO4^{2-}}\\[3mm]
%             &\ce{HPO4^{2-} <=> H+ + PO4^{3-}}
%         \end{tabular}
%     \end{center}

% \newpage

% \subsubsection{电离度和电离常数}
%     我们使用化学式~\ce{AB}表示任意一种一元弱酸或一元弱碱。\\[3mm]
%     我们认为物质~\ce{AB}在水中部分电离的产物为阳离子~\ce{A+}和阴离子~\ce{B-}。\\[3mm]
%     使用化学方程式表示就是:
%     \begin{center}
%         \ce{AB <=> A+ + B-}\\[6mm]
%     \end{center}
%     同时我们做以下符号上的定义:\vspace{5pt}
%     \begin{table}[h]
%         \begin{center}
%             \begin{tabular}{l|l}
%                 \hline
%                 $c_0$\qquad\qquad&电离前溶液中~\ce{AB}的浓度\qquad\qquad\\ \hline
%                 $c(\ce{A+})$\qquad\qquad&电离后溶液中~\ce{A+}的浓度\qquad\qquad\\ \hline
%                 $c(\ce{B-})$\qquad\qquad&电离后溶液中~\ce{B-}的浓度\qquad\qquad\\ \hline
%                 $c(\ce{AB})$\qquad\qquad&电离后溶液中~\ce{AB}的浓度\qquad\qquad\\ \hline
%             \end{tabular}
%             \caption{关于浓度的符号定义}
%         \end{center}
%     \end{table}\\[2mm]
%     我们定义电离度:
%     \begin{large}
%         \begin{equation*}
%             \alpha=\frac{c(\ce{A+})}{c_0}=\frac{c(\ce{B-})}{c_0}
%         \end{equation*}
%     \end{large}\\
%     电离度是一个与溶液浓度有关的量,它衡量了电离的程度大小。\\[8mm]
%     我们定义电离常数:
%     \begin{large}
%         \begin{equation*}
%             K=\frac{c(\ce{A+})\cdot c(\ce{B-})}{c(\ce{AB})}
%         \end{equation*}
%     \end{large}\\
%     电离常数是一个与溶液浓度无关的量,它衡量了物质本身的性质。\\[12mm]
%     接下来我们将推导电离度和电离常数之间的关系。\\[3mm]
%     根据定义我们可以得到:
%     \begin{align}
%         &c(\ce{A+})=c_0\cdot\alpha\\[3mm]
%         &c(\ce{B-})=c_0\cdot\alpha\\[3mm]
%         &c(\ce{AB})=c_0-c_0\cdot\alpha
%     \end{align}

% \newpage

%     代入电离常数的计算式中可以得到:
%     \begin{align}
%         K&=\frac{c(\ce{A+})\cdot c(\ce{B-})}{c(\ce{AB})}\\[4mm]
%         &=\frac{(c_0\cdot\alpha)\cdot(c_0\cdot\alpha)}{c_0-c_0\cdot\alpha}\\[4mm]
%         &=\frac{c_0\,^2\cdot\alpha^2}{c_0\cdot(1-\alpha)}\\[4mm]
%         &=\frac{c_0\cdot\alpha^2}{1-\alpha}
%     \end{align}\\
%     由于弱电解质的电离度很小:
%     \begin{align}
%         1-\alpha\approx 1
%     \end{align}\\
%     电离度和电离常数的关系:
%     \begin{large}
%         \begin{equation*}
%             K=c_0\cdot\alpha^2
%         \end{equation*}
%     \end{large}\\
%     对于多元弱酸和多元弱碱,由于其主要以第一步电离为主,所以该公式也是适用的。\\

% \subsection{离子反应}
%     电解质在水溶液中发生电离,
%     所以电解质在水溶液中的反应是通过离子进行的,
%     我们将这一类反应称为离子反应,
%     离子反应表示的是一类物质间的反应。\\[3mm]
%     离子反应用实际参加反应的离子表示反应过程。\\[6mm]
%     例如盐酸和氢氧化钠的反应:
%     \begin{center}
%         \begin{tabular}{rl}
%             &化学反应方程式:\ce{HCl + NaOH -> NaCl + H2O}\qquad\qquad\qquad \\[3mm]
%             &离子反应方程式:\ce{H+ + OH -> H2O}\\[5mm]            
%         \end{tabular}
%     \end{center}
%     例如氯化钙和碳酸钠反应:
%     \begin{center}
%         \begin{tabular}{rl}
%             &化学反应方程式:\ce{CaCl2 + Na2CO3 -> 2NaCl + CaCO3 v}\\[3mm]
%             &离子反应方程式:\ce{Ca^{2+} + CO3^{2-} -> CaCO3 v}\\[5mm]
%         \end{tabular}
%     \end{center}

% \newpage

% \subsubsection{离子反应的规律}
%     离子反应总是向着离子数量减少的方向进行。\\[3mm]
%     由此我们可以得出离子反应发生的条件:\\[3mm]
%     1.反应有沉淀产生\\[3mm]
%     2.反应有气体产生\\[3mm]
%     3.反应有弱电解质产生\\[5mm]
%     以上三点均可以使得溶液中的离子浓度降低,
%     只需要满足其中一个,该离子反应就可以进行。\\

% \subsection{酸碱度}
%     水是一种极弱的电解质:
%     \begin{center}
%         \ce{H2O <=> H+ + OH-}\\[3mm]
%     \end{center}
%     在25$^\circ$C时,$1$~L纯水,即$55$~mol水分子中只有$1\times10^{-7}$~mol水分子发生电离。\\[3mm]
%     在25$^\circ$C时,水中的氢离子(\ce{H+})和氢氧根离子(\ce{OH-})的浓度均为$1\times10^{-7}$~mol/L。\\[3mm]
%     我们将两者的乘积称为水的离子积:
%     \begin{large}
%         \begin{equation*}
%             \text{K}_\text{w}=c(\ce{H+})\cdot c(\ce{OH-})=1\times 10^{-14}
%         \end{equation*}
%     \end{large}\\
%     在酸碱溶液中,氢离子(\ce{H+})和氢氧根离子(\ce{OH-})均是同时存在的:\\[3mm]
%     酸性溶液中虽然主要以氢离子(\ce{H+})为主,但也存在少量氢氧根离子(\ce{OH-})。\\[3mm]
%     碱性溶液中虽然主要以氢氧根离子(\ce{OH-})为主,但也存在少量氢离子(\ce{H+})。\\[6mm]
%     所以实际上,溶液酸性和碱性的差异,只与氢离子和氢氧根离子的相对浓度有关。\\[3mm]
%     在稀酸溶液和稀碱溶液中,氢离子和氢氧根离子的乘积仍为定值,该定值就是水的离子积。\\[6mm]
%     因此确定一种溶液的酸碱度,我们可以只用氢离子的浓度或氢氧根离子的浓度中的一个来确定。\\[3mm]
%     但是由于两种离子浓度的数值均较小,表示起来并不方便,所以我们进行以下定义。

% \newpage

%     定义pH为溶液中离子~\ce{H+}浓度的负对数:
%     \begin{large}
%         \begin{equation*}
%             \text{pH}=-\lg~c\left(\ce{H+}\right)
%         \end{equation*}
%     \end{large}\\
%     定义pOH为溶液中离子~\ce{OH-}浓度的负对数:
%     \begin{large}
%         \begin{equation*}
%             \text{pOH}=-\lg~c\left(\ce{OH-}\right)
%         \end{equation*}
%     \end{large}\\
%     定义p$\text{K}_\text{w}$为水的离子积p$\text{\text{K}}_\text{w}$的负对数:
%     \begin{large}
%         \begin{equation*}
%             \text{p}\text{K}_\text{w}=-\lg~\text{K}_\text{w}
%         \end{equation*}
%     \end{large}\\
%     当温度等于25$^\circ$C时,p$\text{K}_\text{w}=14$。\\[3mm]
%     当温度大于25$^\circ$C时,水的电离程度更大些,因此p$\text{K}_\text{w}<14$。\\[3mm]
%     当温度小于25$^\circ$C时,水的电离程度更小些,因此p$\text{K}_\text{w}<14$。\\[6mm]
%     三者显然满足以下关系:
%     \begin{large}
%         \begin{equation*}
%             \text{pK}_\text{w}=\text{pH}+\text{pOH}
%         \end{equation*}
%     \end{large}\\
%     溶液为中性时,满足以下关系:
%     \begin{large}
%         \begin{equation*}
%             \text{pH}=\text{pOH}=\frac{\text{pK}_\text{w}}{2}
%         \end{equation*}
%     \end{large}\\
%     溶液为酸性时,满足以下关系:
%     \begin{large}
%         \begin{equation*}
%             \text{pH}<\frac{\text{pK}_\text{w}}{2}\qquad\text{pOH}>\frac{\text{pK}_\text{w}}{2}
%         \end{equation*}
%     \end{large}\\
%     溶液为碱性时,满足以下关系:
%     \begin{large}
%         \begin{equation*}
%             \text{pH}>\frac{\text{pK}_\text{w}}{2}\qquad\text{pOH}<\frac{\text{pK}_\text{w}}{2}
%         \end{equation*}
%     \end{large}\\
%     因此当温度等于25$^\circ$C时,中性溶液pH$=7$,温度升高时该值会降低,温度降低时该值会升高。

% \newpage

%     除了上述的定义外,我们还有一种更为优秀的酸碱度定义。\\[4mm]
%     定义AG为溶液中离子\ce{H+}和离子\ce{OH-}浓度比的对数:\vspace{5pt}
%     \begin{large}
%         \begin{equation*}
%             \text{AG}=\lg\left[~\frac{\text{c(\ce{H+})}}{\text{c(\ce{OH-})}}~\right]
%         \end{equation*}
%     \end{large}\\
%     溶液为中性时,满足以下关系:
%     \begin{large}
%         \begin{equation*}
%             \text{AG}=0
%         \end{equation*}
%     \end{large}\\
%     溶液为酸性时,满足以下关系:
%     \begin{large}
%         \begin{equation*}
%             \text{AG}>0
%         \end{equation*}
%     \end{large}\\
%     溶液为碱性时,满足以下关系:
%     \begin{large}
%         \begin{equation*}
%             \text{AG}<0
%         \end{equation*}
%     \end{large}\\
%     使用酸度AG表示酸碱度有以下几个优势:\\[3mm]
%     1.通过定义,避免了对有符号数取对数,从数学上看更为严谨。\\[3mm]
%     2.在任何温度下中性溶液的值均为零,便于酸性和碱性的直观判断。\\[6mm]
%     根据酸度AG的定义:
%     \setcounter{equation}{0}
%     \begin{align}
%         \text{AG}
%         &=\lg\left[~\frac{\text{c(\ce{H+})}}{\text{c(\ce{OH-})}}~\right]\\[4mm]
%         &=\lg~c\left(\ce{H+}\right)-\lg~c\left(\ce{OH-}\right)\\[4mm]
%         &=\text{pOH}-\text{pH}
%     \end{align}\\
%     酸度AG的换算关系:
%     \begin{large}
%         \begin{align*}
%             &\text{AG}=\text{pOH}-\text{pH}\\[3mm]
%             &\text{AG}=\text{pK}_\text{W}-2\cdot\text{pH}\\[3mm]
%         \end{align*}
%     \end{large}

% \newpage

% \subsection{水解}
%     在盐的水溶液中,
%     盐所电离出的阳离子或阴离子与
%     水所电离出的氢氧根离子或氢离子相互结合,
%     形成弱酸或弱碱的反应,
%     称为盐类的水解反应。\\[3mm]
%     提高温度,稀释溶液,均可以促进水解。

% \subsubsection{强酸强碱盐的水解}
%     强酸强碱盐不发生水解,所以其水溶液成中性。

% \subsubsection{强酸弱碱盐的水解}
%     强酸弱碱盐会发生水解,水解过程中消耗了氢氧根离子,所以其水溶液呈酸性。
%     \begin{center}
%         \ce{Fe^{2+} + 2H2O <=> Fe(OH)2 + 2H+}\\[3mm]
%         \ce{Cu^{2+} + 2H2O <=> Cu(OH)2 + 2H+}
%     \end{center}

% \subsubsection{弱酸强碱盐的水解}
%     弱酸强碱盐会发生水解,水解过程中消耗了氢离子,其水溶液呈碱性。\\[3mm]
%     弱酸酸根的水解是分布进行的,且以第一步为主。
%     \begin{center}
%         \begin{tabular}{rl}
%             &\ce{CO3^{2-} + H2O <=> HCO3- + OH-}\\[2mm]
%             &\ce{HCO3- + H2O <=> CO3^{2-} + OH-}\\[5mm]
%             &\ce{PO4^{3-} + H2O <=> HPO4^{2-} + OH-}\\[2mm]
%             &\ce{HPO4^{2-} + H2O <=> H2PO4^{-} + OH-}\\[2mm]
%             &\ce{H2PO4^{-} + H2O <=> H3PO4 + OH-}\\[2mm]
%         \end{tabular}
%     \end{center}

% \subsubsection{弱酸弱碱盐的水解}
%     弱酸弱碱盐会发生水解,而且通常较易发生水解,
%     其酸碱性需要依靠其电离程度判断。\\[3mm]
%     若酸的电离程度强于碱,显酸性。\\[2mm]
%     若碱的电离程度强于酸,显碱性。\\[2mm]
%     若酸和碱的电离程度相近,显中性。\\[4mm]
%     醋酸铵(NH4Ac)的水溶液呈中性,其水解方程式:
%     \begin{center}
%         \begin{tabular}{rl}
%             &\ce{NH4+ + Ac- + H2O <=> NH3\cdot H2O + HAc}\\[3mm]
%         \end{tabular}
%     \end{center}

% \newpage

% \subsubsection{强酸酸式盐的水解}
%     强酸酸式盐不发生水解,由于可以电离出氢离子,所以其水溶液成酸性。

% \subsubsection{弱酸酸式盐的水解}
%     强酸酸式盐会同时发生水解和电离,酸碱性取决于电离程度和水解程度的相对强弱。\\[3mm]
%     大部分酸式盐的水解强于电离,此时产生的氢氧根离子更多,呈碱性。\\[2mm]
%     少部分酸式盐的电离强于水解,此时产生的氢离子更多,呈酸性。\\[3mm]
%     碳酸氢根的电离弱于水解,呈碱性:
%     \begin{center}
%         \begin{tabular}{rl}
%             &电离[弱]:\ce{HCO3^{-} <=> CO3^{2-} + H+}\\[3mm]
%             &水解[强]:\ce{HCO3^{-} + H2O <=> H2CO4 + OH-}\\[3mm]
%         \end{tabular}
%     \end{center}
%     亚硫酸氢根的电离强于水解,呈酸性:
%     \begin{center}
%         \begin{tabular}{rl}
%             &电离[强]:\ce{HSO3- <=> SO3^{2-} + H+}\\[3mm]
%             &水解[弱]:\ce{HSO3- + H2O <=> H2SO3 + OH-}\\[3mm]
%         \end{tabular}
%     \end{center}
%     磷酸二氢根的电离强于水解,呈酸性:
%     \begin{center}
%         \begin{tabular}{rl}
%             &电离[强]:\ce{H2PO4- <=> HPO4^{2-} + H+}\\[3mm]
%             &水解[弱]:\ce{H2PO4- + H2O <=> H3PO4 + OH-}\\[3mm]
%         \end{tabular}
%     \end{center}
%     实际上,只有亚硫酸氢根(\ce{HSO3-})和磷酸二氢根(\ce{H2PO4-})所构成的酸式盐为酸性,
%     绝大多数的酸式盐的水解强于电离,呈碱性。

% \newpage

% \section{电化学}

% \subsection{原电池}
%     原电池是一种将化学能转换为电能的装置。\\[3mm]
%     原电池由正极,负极,电解质溶液组成,
%     原电池的负极通常为活泼性较强的金属,
%     原电池的正极通常为活泼性较差的金属,
%     也可以是能导电的非金属。\\[3mm]
%     原电池的本质是活泼金属与电解质溶液的氧化还原反应,
%     然而通过原电池,氧化还原反应被拆分为两个半反应,
%     分别在负极和正极进行。\\[2mm]
%     负极发生氧化反应,负极金属单质失去电子,被氧化为对应离子。\\[1mm]
%     正极发生还原反应,溶液中阳离子得到电子,被还原为对应单质。\\[3mm]
%     原电池中,只有负极和电解质溶液参与化学反应,正极只起到了导体的作用,并不参与反应。\\[4mm]
%     请考虑以下两组例子:\\[3mm]
%     1.电解质溶液:\ce{H2SO4}\qquad 负极:\ce{Zn}\qquad 正极:\ce{Cu}\\[3mm]
%     负极氧化反应(现象:质量减轻):
%     \begin{center}
%         \ce{Zn - 2e -> Zn^{2+}}\\[3mm]    
%     \end{center}
%     正极还原反应(现象:产生气体):
%     \begin{center}
%         \ce{2H+ + 2e -> H2 ^}\\[3mm]    
%     \end{center}
%     总反应方程式:
%     \begin{center}
%         \ce{Zn + H2SO4 -> ZnSO4 + H2 ^}\\[8mm]
%     \end{center}
%     2.电解质溶液:\ce{CuSO4}\qquad 负极:\ce{Zn}\qquad 正极:\ce{Cu}\\[3mm]
%     负极氧化反应(现象:质量减轻):
%     \begin{center}
%         \ce{Zn - 2e -> Zn^{2+}}\\[3mm]    
%     \end{center}
%     正极还原反应(现象:质量增大):
%     \begin{center}
%         \ce{Cu^{2+} + 2e -> Cu}\\[3mm]    
%     \end{center}
%     总反应方程式:
%     \begin{center}
%         \ce{Zn + CuSO4 -> ZnSO4 + Cu}
%     \end{center}

% \newpage

% \subsection{电解池}
%     电解池是一种将电能转化为化学能的装置。\\[3mm]
%     电解池由阳极,阴极,电解质溶液组成,
%     阳极接入电池正极,阴极接入电池负极,
%     通电后,电解质溶液将在电流的作用下发生电解,
%     阳极和阴极一般情况下不参与电解反应。\\[3mm]
%     电解池是用于进行电解的装置,
%     电解的本质是电解质溶液在电流的作用下,
%     发生氧化还原反应的过程,
%     该氧化还原反应被拆分为两个半反应,
%     分别在阳极和阴极进行。\\[2mm]
%     阳极发生氧化反应,溶液中阴离子失去电子,被氧化为对应单质。\\[1mm]
%     阴极发生还原反应,溶液中阳离子得到电子,被还原为对应单质。\\[3mm]  
%     在电解过程中,需要注意放电顺序:\\[2mm]
%     阴离子放电顺序:\ce{S^{2-}}~~>~~\ce{I^-}~~>~~\ce{Br-}~~>~~\ce{Cl-}~~>~~\ce{OH-}~~>~~含氧酸根离子\\[2mm]
%     阳离子放电顺序:\ce{Ag+}~~>~~\ce{Fe^{3+}}~~>~~\ce{Cu^{2+}}~~>~~\ce{H^{+}}~~>~~\ce{Fe^{2+}}~~>~~\ce{Al^{3+}}~~>~~\ce{Mg^{2+}}~~>~~\ce{Na^{+}}~~>~~\ce{Ca^{2+}}~~>~~\ce{K+}\\[4mm]
%     请考虑以下两组例子:\\[3mm]
%     1.电解质溶液:\ce{CuCl2}\\[3mm]
%     阳极发生氧化反应(阴离子放电顺序:\ce{Cl-}~~>~~\ce{OH-}):
%     \begin{center}
%         \ce{2Cl- - 2e -> Cl2 ^}\\[5mm]
%     \end{center}
%     阴极发生还原反应(阳离子放电顺序:\ce{Cu^{2+}}~~>~~\ce{H+}):
%     \begin{center}
%         \ce{Cu^{2+} + 2e -> Cu}\\[3mm]
%     \end{center}
%     总反应方程式:
%     \begin{center}
%         \ce{CuCl2 -> Cu + Cl2 ^}\\[6mm]
%     \end{center}
%     2.电解质溶液:\ce{NaCl}\\[3mm]
%     阳极发生氧化反应(阴离子放电顺序:\ce{Cl-}~~>~~\ce{OH-}):
%     \begin{center}
%         \ce{2Cl- - 2e -> Cl2 ^}\\[5mm]
%     \end{center}
%     阴极发生还原反应(阳离子放电顺序:\ce{H^{+}}~~>~~\ce{Na^{+}}):
%     \begin{center}
%         \ce{2H+ + 2e -> H2 ^}\\[3mm]
%     \end{center}
%     总反应方程式:
%     \begin{center}
%         \ce{2NaCl + H2O -> H2 ^ + Cl2 ^ + NaOH}\\[5mm]
%     \end{center}
%     由于氢离子随着电解逐渐变为氢气溢出,
%     导致氢氧根离子在阴极大量聚集,所以最终的结果是阳极产生氯气,
%     阴极产生氢气,同时也产生氢氧化钠。

\newpage

\section{烷~~烯~~炔}

\subsection{有机物}
    我们将含有碳元素的化合物称为有机物,研究有机物的化学称为有机化学。\\[3mm]
    由于一些历史原因,碳酸盐,碳氧化物,氰化物,这些物质虽然含碳但仍然被归入无机物。\vspace{3pt}

\subsubsection{有机物的突破}
    在1800年以前,科学家已经可以从天然的动植物中分离提纯某些有机物,其中大多数用于医药,
    科学家在用人工方法合成有机物的实验屡遭失败之后,当时人们普遍认为有机物不同于无机物,
    无法通过人工合成,只能通过动植物体内神秘的生命力的控制下产生。\\[3mm]
    在1824年德国年轻化学家维勒,通过煮沸含有铵根离子和氰酸根离子的水溶液制取氰酸铵时,
    意外的发现制取的白色晶体,并不是无机物氰酸铵,而是有机物尿素。\\[3mm]
    在加热条件下氰酸铵分子结构发生重排,由氰酸铵变为了尿素:\vspace{5pt}
    \begin{center}
        \chemnameinit{\chemfig{NH_4CNO}}
        \setchemfig{atom sep=15pt}
        \setchemfig{fixed length=true}
        \setchemfig{arrow coeff=0.8}
        
        \schemestart
            \chemname[8pt]{\chemfig{NH_4CNO}}{氰酸铵}
            ~~~~\arrow(foo.mid east--bar.mid west){->[\footnotesize 加热]}~~~~
            \chemname[8pt]{\chemfig{C(-[0]NH_2)(-[4]NH_2)(=[2]O)}}{尿素}
        \schemestop
    \end{center}\vspace{10pt}
    维勒的实验第一次通过人工方法,由无机物制取有机物。\\[3mm]
    维勒的实验强烈的震撼了生命力说,突破了无机物和有机物的鸿沟。\vspace{3pt}

\subsubsection{有机物的特点}
    有机物中的碳是一种非常特别的元素,碳碳键在碳氢化合物中可以稳定的存在,形成分子骨架。\\[3mm]
    有机物中的碳氢化合物在部分的替换为其他的非金属元素后,仍然能保持稳定的分子骨架。\\[6mm]
    下表列出了有机物中各个元素化合价:\vspace{5pt}
    \begin{table}[h]
        \begin{center}
            \begin{tabular}{l|l|l}
                \hline
                碳元素~~~~~~~~~~~~&\ce{C}~~~~~~~~~~~~&$+4$价~~~~~~~~~~~~\\ \hline
                氮元素~~~~~~~~~~~~&\ce{N}~~~~~~~~~~~~&$+3$价~~~~~~~~~~~~\\ \hline
                氧元素~~~~~~~~~~~~&\ce{O}~~~~~~~~~~~~&$+2$价~~~~~~~~~~~~\\ \hline
                硫元素~~~~~~~~~~~~&\ce{S}~~~~~~~~~~~~&$+2$价~~~~~~~~~~~~\\ \hline
                氢元素~~~~~~~~~~~~&\ce{H}~~~~~~~~~~~~&$+1$价~~~~~~~~~~~~\\ \hline
            \end{tabular}
            \caption{有机物中各个元素的化合价}
        \end{center}
    \end{table}\\
    除此之外,卤族元素在有机物中表现为$+1$价。

\newpage

\subsection{烷}
    烷指的是一类碳原子间均为碳碳单键的碳氢化合物。

\subsubsection{甲烷}
    甲烷(\ce{CH4})是最简单的烷烃,结构式为:\vspace{5pt}
    \begin{center}
        \chemnameinit{\chemfig{C(-[0]H)(-[2]H)(-[4]H)(-[6]H)}}
        \setchemfig{atom sep=15pt}
        \setchemfig{fixed length=true}
        \chemname[5pt]
        {
            \chemfig
            {
                C(-[0]H)(-[2]H)(-[4]H)(-[6]H)
            }
        }{甲烷}
    \end{center}\vspace{10pt}
    甲烷可以通过无水醋酸钠和碱石灰混合加热制取:
    \begin{center}
        \ce{CH3COONa + NaOH ->T[\ce{CaO}][加热] Na2CO3 + CH4 ^}\\[3mm]
    \end{center}
    甲烷可以在空气中燃烧,火焰颜色为淡蓝色:
    \begin{center}
        \ce{CH4 + 2O2 ->T[点燃] CO2 + 2H2O}\\[4mm]
    \end{center}
    有机物的取代反应:有机物分子中的某些原子团被其他原子团替代的反应。\\[3mm]
    甲烷可以和氯气发生取代反应:\vspace{15pt}
    \begin{center}
        \chemnameinit{}
        \setchemfig{atom sep=15pt}
        \setchemfig{fixed length=true}
        \setchemfig{arrow coeff=0.8}

        \schemestart
            \chemfig{C(-[0]H)(-[2]H)(-[4]H)(-[6]H)}
            \+{12pt,10pt,1pt}
            \ce{Cl2}
            ~~~~\arrow(foo.mid east--bar.mid west){->[\footnotesize 光照]}~~~~
            \chemname[5pt]{\chemfig{C(-[0]Cl)(-[2]H)(-[4]H)(-[6]H)}}{一氯甲烷}
            \+{12pt,10pt,1pt}
            \ce{HCl}
        \schemestop\vspace{14pt}

        \schemestart
            \chemfig{C(-[0]Cl)(-[2]H)(-[4]H)(-[6]H)}
            \+{12pt,10pt,1pt}
            \ce{Cl2}
            ~~~~\arrow(foo.mid east--bar.mid west){->[\footnotesize 光照]}~~~~
            \chemname[5pt]{\chemfig{C(-[0]Cl)(-[2]H)(-[4]H)(-[6]Cl)}}{二氯甲烷}
            \+{12pt,10pt,1pt}
            \ce{HCl}
        \schemestop\vspace{14pt}

        \schemestart
            \chemfig{C(-[0]Cl)(-[2]H)(-[4]H)(-[6]Cl)}
            \+{12pt,10pt,1pt}
            \ce{Cl2}
            ~~~~\arrow(foo.mid east--bar.mid west){->[\footnotesize 光照]}~~~~
            \chemname[5pt]{\chemfig{C(-[0]Cl)(-[2]Cl)(-[4]H)(-[6]Cl)}}{三氯甲烷}
            \+{12pt,10pt,1pt}
            \ce{HCl}
        \schemestop\vspace{14pt}

        \schemestart
            \chemfig{C(-[0]Cl)(-[2]Cl)(-[4]H)(-[6]Cl)}
            \+{12pt,10pt,1pt}
            \ce{Cl2}
            ~~~~\arrow(foo.mid east--bar.mid west){->[\footnotesize 光照]}~~~~
            \chemname[5pt]{\chemfig{C(-[0]Cl)(-[2]Cl)(-[4]Cl)(-[6]Cl)}}{四氯甲烷}
            \+{12pt,10pt,1pt}
            \ce{HCl}
        \schemestop
    \end{center}

\newpage

\subsubsection{烷烃}
    烷烃指的是一类碳原子间均为碳碳单键的链状碳氢化合物。\\[3mm]
    烷烃的分子式可以使用通式~\ce{C_{n}H_{2n+2}}表示。\\[4mm]
    以下列出了一些常见的烷烃:\vspace{5pt}
    \begin{center}
        \chemnameinit{\chemfig{C(-[0]H)(-[2]H)(-[4]H)(-[6]H)}}
        \setchemfig{atom sep=15pt}
        \setchemfig{fixed length=true}
        \chemname[5pt]
        {
            \chemfig
            {
                C(-[0]H)(-[2]H)(-[4]H)(-[6]H)
            }
        }{甲烷}\qquad
        \chemname[5pt]
        {
            \chemfig
            {
                C(-[2]H)(-[4]H)(-[6]H)-C(-[2]H)(-[0]H)(-[6]H)
            }
        }{乙烷}\qquad
        \chemname[5pt]
        {
            \chemfig
            {
                C(-[2]H)(-[4]H)(-[6]H)-C(-[2]H)(-[6]H)-C(-[2]H)(-[0]H)(-[6]H)
            }
        }{丙烷}
    \end{center}\vspace{10pt}
    烷烃可以根据所含碳原子的数目进行命名:\\[3mm]
    数量小于$10$的使用天干命名:\ruby{甲}{ji\v a}烷,\ruby{乙}{y\v i}烷,\ruby{丙}{b\v ing}烷,\ruby{丁}{d\= ing}烷,\ruby{戊}{w\` u}烷,\ruby{己}{j\v i}烷,\ruby{庚}{g\= eng}烷,\ruby{辛}{x\= in}烷,\ruby{壬}{r\' en}烷,\ruby{癸}{gu\v i}烷。\\[3mm]
    数量大于$10$的使用数字命名:十一烷,十二烷,十三烷,十四烷,二十烷,五十烷,一百烷。\\[6mm]
    烷烃的碳原子数目越多,结合方式越复杂,同分异构体的数量越多。\\[3mm]
    烷烃中丙烷只有一种同分异构体,丁烷有两种同分异构体,戊烷有三种同分异构体。\\[6mm]
    通常用异和新表示含有如下结构的烷烃:\vspace{5pt}
    \begin{center}
        \chemnameinit{\chemfig{C(-[0])(-[2]CH_3)(-[4]CH_3)(-[6]CH_3)}}
        \setchemfig{atom sep=15pt}
        \setchemfig{fixed length=true}

        \chemname[5pt]
        {
            \chemfig
            {
                CH(-[0])(-[3]CH_3)(-[5]CH_3)
            }
        }{异}\qquad\qquad
        \chemname[5pt]
        {
            \chemfig
            {
                C(-[0])(-[2]CH_3)(-[4]CH_3)(-[6]CH_3)
            }
        }{新}
    \end{center}\vspace{15pt}
    以下列出了丁烷的两种同分异构体:\vspace{5pt}
    \begin{center}
        \chemnameinit{\chemfig{CH_3-CH(-[6]CH_3)-CH_3}}
        \setchemfig{atom sep=15pt}
        \setchemfig{fixed length=true}
        \chemname[5pt]
        {
            \chemfig
            {
                CH_3-CH_2-CH_2-CH_3
            }
        }{正丁烷}\qquad\quad
        \chemname[5pt]
        {
            \chemfig
            {
                CH_3-CH(-[6]CH_3)-CH_3
            }
        }{异丁烷}
    \end{center}\vspace{15pt}
    以下列出了戊烷的三种同分异构体:\vspace{-5pt}
    \begin{center}
        \chemnameinit{\chemfig{CH_3-C(-[2]CH3)(-[6]CH3)-CH_3}}
        \setchemfig{atom sep=15pt}
        \setchemfig{fixed length=true}
        \chemname[5pt]
        {
            \chemfig
            {
                CH_3-CH_2-CH_2-CH_2-CH_3
            }
        }{正戊烷}\qquad\quad
        \chemname[5pt]
        {
            \chemfig
            {
                CH_3-CH(-[6]CH_3)-CH_2-CH_3
            }
        }{异戊烷}\qquad\quad
        \chemname[5pt]
        {
            \chemfig
            {
                CH_3-C(-[2]CH3)(-[6]CH3)-CH_3
            }
        }{新戊烷}
    \end{center}

\newpage

    烷烃的系统命名法:\\[4mm]
    1.主链的选择\\[3mm]
    命名烷烃时需要选择一条碳链作为主链,其他的支链视为主链的取代基。\\[3mm]
    首先考虑主链的碳原子数量,最长的一条作为主链。\\[3mm]
    其次考虑主链上侧链的数量,最多的一条作为主链。\\[6mm]
    2.主链的编号\\[3mm]
    命名烷烃时需要对选定的主链进行编号,便于后续标记支链的位置。\\[3mm]
    主链的编号依据最低系列原则,选取编号方式中使得取代基的位号序列字典序最小的一种。\\[6mm]
    3.主链的命名\\[3mm]
    设主链的碳原子数量为X,将主链命名为:
    \begin{center}
        X烷~~~~~~~~X\,$\in\big\{\text{甲},\text{乙},\text{丙},\text{丁},\cdots\big\}$
    \end{center}\vspace{10pt}
    4.侧链的命名\\[3mm]
    设侧链的数量为N,设侧链的位号位$\text{A}_1\sim \text{A}_n$,将侧链命名为:\vspace{2pt}
    \begin{center}
        $\text{A}_1,\text{A}_2,\text{A}_n$\,-\,N取代基~~~~~~~~N\,$\in\big\{\text{一},\text{二},\text{三},\text{四},\cdots\big\}$~~~~~~~~$\text{A}_1\sim \text{A}_n\in\big\{1,2,3,4,\cdots\big\}$
    \end{center}\vspace{15pt}
    5.完成命名\\[3mm]
    将侧链按照简单到复杂的顺序依次用减号连接,添加在主链前:
    \begin{center}
        [侧链$1$]\,-\,[侧链$2$]\,-\,[侧链Y][主链]
    \end{center}\vspace{25pt}
    以下列出了部分使用系统命名法的烷烃:\vspace{5pt}
    \begin{center}
        \chemnameinit{\chemfig{CH_3-CH(-[6]CH_3)-CH_2-CH(-[6]CH_3-[6]CH_3)-CH_3}}
        \setchemfig{atom sep=15pt}
        \setchemfig{fixed length=true}
        \chemname[5pt]
        {
            \chemfig
            {
                CH_3-CH(-[6]CH_3)-CH_2-CH(-[6]CH_2-[6]CH_3)-CH_3
            }
        }{2,4-二甲基己烷}\qquad\quad
        \chemname[5pt]
        {
            \chemfig
            {
                CH_3-CH_2-CH(-[6]CH_3)-CH_2-CH(-[6]CH_2-[6]CH_3)-CH(-[6]CH_3)-CH_3
            }
        }{2,5-二甲基-3-乙基庚烷}
    \end{center}
    
\newpage

\subsubsection{环烷烃}
    环烷烃指的是一类碳原子间均为碳碳单键的环状碳氢化合物。\\[3mm]
    环烷烃的分子式可以使用通式~\ce{C_{n}H_{2n}}表示。\\[4mm]
    以下列出了一些常见的环烷烃:\vspace{5pt}
    \begin{center}
        \chemnameinit{\chemfig{[:30]C*6(-C(-[:-30]H)(-[:-90]H)-C(-[:30]H)(-[:-30]H)-C(-[:90]H)(-[:30]H)-C(-[:150]H)(-[:90]H)-C(-[:210]H)(-[:150]H)-)(-[:-150]H)(-[:-90]H)}}
        \setchemfig{atom sep=25pt}
        \chemname[15pt]
        {
            \chemfig
            {
                [:18]C*5(-C(-[:-36]H)(-[:-72]H)-C(-[:0]H)(-[:36]H)-C(-[:72]H)(-[:108]H)-C(-[:180]H)(-[:144]H)-)(-[:-144]H)(-[:-108]H)
            }
        }{环戊烷}\qquad\qquad
        \chemname[15pt]
        {
            \chemfig
            {
                [:30]C*6(-C(-[:-30]H)(-[:-90]H)-C(-[:30]H)(-[:-30]H)-C(-[:90]H)(-[:30]H)-C(-[:150]H)(-[:90]H)-C(-[:210]H)(-[:150]H)-)(-[:-150]H)(-[:-90]H)
            }
        }{环己烷}\qquad
    \end{center}\vspace{20pt}
    以下列出了一些常见的环烷烃的简写:\vspace{5pt}
    \begin{center}
        \chemnameinit{\chemfig{[:30]C*6(------)}}
        \setchemfig{atom sep=25pt}
        \chemname[15pt]
        {
            \chemfig
            {
                [:18]*5(-----)
            }
        }{环戊烷}\qquad\qquad\quad
        \chemname[15pt]
        {
            \chemfig
            {
                [:30]*6(------)
            }
        }{环己烷}\qquad
    \end{center}\vspace{15pt}
    环烷烃对碳原子数量最低要求是三个,即环丙烷是最简单的环烷烃。\\[3mm]
    环烷烃的化学性质和对应的链烷烃的化学性质基本相似。

\newpage

\subsection{烯}
    烯指的是一类碳原子间存在碳碳双键的碳氢化合物。

\subsubsection{乙烯}
    乙烯(\ce{C2H4})是最简单的烯烃,结构式为:\vspace{5pt}
    \begin{center}
        \chemnameinit{\chemfig{C(-[4]H)(-[6]H)=C(-[6]H)(-[0]H)}}
        \setchemfig{atom sep=15pt}
        \setchemfig{fixed length=true}
        \chemname[5pt]
        {
            \chemfig
            {
                C(-[4]H)(-[6]H)=C(-[6]H)(-[0]H)
            }
        }{乙烯}
    \end{center}\vspace{10pt}
    乙烯可以通过酒精和浓硫酸混合加热制取:
    \begin{center}
        \ce{C2H5OH ->T[\text{浓}\ce{H2SO4}][加热] H2O + C2H4 ^}\\[3mm]
    \end{center}
    乙烯可以在空气中燃烧,火焰明亮伴有黑烟:
    \begin{center}
        \ce{C2H4 + 3O2 ->T[点燃] 2CO2 + 2H2O}\\[4mm]
    \end{center}
    有机物的加成反应:有机物分子中不饱和碳原子通过断开部分键与其他原子团直接结合的反应。\\[3mm]
    乙烯可以和溴水发生加成反应:\vspace{5pt}
    \begin{center}
        \chemnameinit{\chemfig{C(-[2]H)(-[4]Br)(-[6]H)-C(-[2]H)(-[0]Br)(-[6]H)}}
        \setchemfig{atom sep=15pt}
        \setchemfig{fixed length=true}
        \setchemfig{arrow coeff=0.8}

        \schemestart
            \chemfig{C(-[4]H)(-[6]H)=C(-[6]H)(-[0]H)}
            \+{12pt,10pt,1pt}
            \ce{Br2}
            ~~~~\arrow(foo.mid east--bar.mid west){->}~~~~
            \chemname[5pt]{\chemfig{C(-[2]Br)(-[4]H)(-[6]H)-C(-[2]Br)(-[0]H)(-[6]H)}}{1,2-二溴乙烷}
        \schemestop
    \end{center}\vspace{15pt}
    乙烯可以通过和氢气的加成反应生成乙烷:
    \begin{center}
        \chemnameinit{\chemfig{C(-[2]H)(-[4]Br)(-[6]H)-C(-[2]H)(-[0]Br)(-[6]H)}}
        \setchemfig{atom sep=15pt}
        \setchemfig{fixed length=true}
        \setchemfig{arrow coeff=0.8}

        \schemestart
            \chemname[5pt]{\chemfig{C(-[4]H)(-[6]H)=C(-[6]H)(-[0]H)}}{乙烯}
            \+{12pt,10pt,1pt}
            \ce{H2}
            ~~~~~\arrow(foo.mid east--bar.mid west){->[\footnotesize 催化剂][\footnotesize 加热]}~~~~
            \chemname[5pt]{\chemfig{C(-[2]H)(-[4]H)(-[6]H)-C(-[2]H)(-[0]H)(-[6]H)}}{乙烷}
        \schemestop
    \end{center}\vspace{15pt}
    乙烯可以通过和氯化氢的加成反应生成氯乙烷:
    \begin{center}
        \chemnameinit{\chemfig{C(-[2]H)(-[4]Br)(-[6]H)-C(-[2]H)(-[0]Br)(-[6]H)}}
        \setchemfig{atom sep=15pt}
        \setchemfig{fixed length=true}
        \setchemfig{arrow coeff=0.8}

        \schemestart
            \chemname[5pt]{\chemfig{C(-[4]H)(-[6]H)=C(-[6]H)(-[0]H)}}{乙烯}
            \+{12pt,10pt,1pt}
            \ce{HCl}
            ~~~\arrow(foo.mid east--bar.mid west){->[\footnotesize 催化剂][\footnotesize 加热]}~~~~
            \chemname[5pt]{\chemfig{C(-[2]H)(-[4]H)(-[6]H)-C(-[2]H)(-[0]H)(-[6]Cl)}}{氯乙烷}
        \schemestop
    \end{center}\vspace{15pt}

\newpage

    有机物的聚合反应:通过相对分子量较小的有机物的互相结合,形成相对分子量很大的有机物。\\[3mm]
    乙烯可以通过聚合反应生成聚乙烯:
    \begin{center}
        \setchemfig{atom sep=15pt}
        \setchemfig{fixed length=true}
        \setchemfig{arrow coeff=0.8}

        \schemestart
            \chemfig{nCH_2=CH_2}
            ~~~~~\arrow(foo.mid east--bar.mid west){->[\footnotesize 催化剂]}~~~~
            \chemfig{\text{\big[}-CH_2-CH_2-\text{\big]$_n$}}
        \schemestop
    \end{center}

\subsubsection{烯烃}
    烯烃指的是一类碳原子间存在一个碳碳双键的链状碳氢化合物。\\[3mm]
    烯烃的分子式可以使用通式\ce{C_{n}H_{2n}}表示。\\[3mm]
    以下列出了一些常见的烯烃:\vspace{5pt}
    \begin{center}
        \chemnameinit{\chemfig{C(-[4]H)(-[6]H)=C(-[6]H)(-[0]H)}}
        \setchemfig{atom sep=15pt}
        \setchemfig{fixed length=true}
        \chemname[5pt]
        {
            \chemfig
            {
                C(-[4]H)(-[6]H)=C(-[6]H)(-[0]H)
            }
        }{乙烯}\qquad\qquad
        \chemname[5pt]
        {
            \chemfig
            {
                C(-[4]H)(-[6]H)=C(-[6]H)-C(-[6]H)(-[2]H)(-[0]H)
            }
        }{丙烯}
    \end{center}\vspace{10pt}
    烯烃的系统命名法:\\[3mm]
    在主链的选择时,还要首先保证选择的主链中含有双键。\\[3mm]
    在主链的编号时,还要首先保证编号的起点离双键最近。\\[3mm]
    设主链的碳原子数量为X,设双键连接的两个碳原子中较小的编号为B,将主链命名为:
    \begin{center}
        B-X烯~~~~~~~~X\,$\in\big\{\text{甲},\text{乙},\text{丙},\text{丁},\cdots\big\}$~~~~~~~~$\text{B}\in\big\{1,2,3,4,\cdots\big\}$\\[5mm]
    \end{center}
    在完成命名时,需要在主链的名称前添加减号。\\[8mm]
    以下列出了部分使用系统命名法的烯烃:\vspace{5pt}
    \begin{center}
        \chemnameinit{\chemfig{CH_2=CH-CH_2=CH_3}}
        \setchemfig{atom sep=15pt}
        \setchemfig{fixed length=true}
        \chemname[15pt]
        {
            \chemfig
            {
                CH_2=CH-CH_2-CH_3
            }
        }{1-丁烯}\qquad\qquad
        \chemname[15pt]
        {
            \chemfig
            {
                CH_3-CH=CH-CH_3
            }
        }{2-丁烯}
    \end{center}\vspace{20pt}
    \begin{center}
        \chemnameinit{\chemfig{CH_3-CH(-[6]CH_3)-CH_2-CH(-[6]CH_3)-CH_2-CH(-[6]CH_2-[6]CH_3)=CH_2}}
        \setchemfig{atom sep=15pt}
        \setchemfig{fixed length=true}
        \chemname[5pt]
        {
            \chemfig
            {
                CH_3-CH_2-CH(-[6]CH_3)-CH_2-CH(-[6]CH_3)-CH_2-C(-[6]CH_2-[6]CH_3)=CH_2
            }
        }{4,6-二甲基-2-乙基-1-辛烯}
    \end{center}

\subsubsection{二烯烃}
    二烯烃指的是一类碳原子间存在两个碳碳双键的链状碳氢化合物。\\[3mm]
    二烯烃的分子式可以使用通式\ce{C_{n}H_{2n}}表示。\\[6mm]
    二烯烃的系统命名法:\\[3mm]
    在主链的选择时,还要首先保证选择的主链中含有两个双键。\\[3mm]
    在主链的编号时,还要首先保证编号的起点离任意双键最近。\\[3mm]
    设主链的碳原子数量为X,设双键连接的两个碳原子中较小的编号为B$_1$$\sim$B$_2$,将主链命名为:
    \begin{center}
        B$_1$,B$_2$-X二烯~~~~~~~~X\,$\in\big\{\text{甲},\text{乙},\text{丙},\text{丁},\cdots\big\}$~~~~~~~~$\text{B}_1\sim\text{B}_2\in\big\{1,2,3,4,\cdots\big\}$\\[5mm]
    \end{center}
    在完成命名时,需要在主链的名称前添加减号。\\[8mm]
    以下列出了部分使用系统命名法的烯烃:\vspace{5pt}
    \begin{center}
        \chemnameinit{\chemfig{CH_2=C(-[6]CH_3)-CH=CH_2}}
        \setchemfig{atom sep=15pt}
        \setchemfig{fixed length=true}
        \chemname[4pt]
        {
            \chemfig
            {
                CH_2=C=CH-CH_3
            }
        }{1,2-丁二烯}\qquad\quad
        \chemname[5pt]
        {
            \chemfig
            {
                CH_2=CH-CH=CH_2
            }
        }{1,3-丁二烯}\qquad\quad
        \chemname[5pt]
        {
            \chemfig
            {
                CH_2=C(-[6]CH_3)-CH=CH_2
            }
        }{2-甲基-1,3-丁二烯}
    \end{center}\vspace{20pt}
    异戊二烯可以通过聚合反应生成聚异戊二烯:\vspace{5pt}
    \begin{center}
        \setchemfig{atom sep=15pt}
        \setchemfig{fixed length=true}
        \setchemfig{arrow coeff=0.8}

        \schemestart
            \chemfig{nCH_2=C(-[6]CH_3)-CH=CH_2}
            ~~~~~\arrow(foo.mid east--bar.mid west){->[\footnotesize 催化剂]}~~~~
            \chemfig{\text{\big[}-CH_2-C(-[6]CH_3)=CH-CH_2-\text{\big]$_n$}}
        \schemestop
    \end{center}\vspace{10pt}
    反应生成的聚异戊二烯实际上就是天然橡胶的主要成分。\\[8mm]
    累积二烯烃:二烯烃中的两个双键间隔的单键数目$=0$\\[3mm]
    共轭二烯烃:二烯烃中的两个双键间隔的单键数目$=1$\\[3mm]
    隔离二烯烃:二烯烃中的两个双键间隔的单键数目$\ge 2$

\newpage

\subsection{炔}
    炔指的是一类碳原子间存在碳碳叁键的碳氢化合物。

\subsubsection{乙炔}
    乙炔(\ce{C2H2})是最简单的炔烃,结构式为:\vspace{5pt}
    \begin{center}
        \chemnameinit{\chemfig{H-C~C-H}}
        \setchemfig{atom sep=15pt}
        \setchemfig{fixed length=true}
        \chemname[15pt]
        {
            \chemfig
            {
                H-C~C-H
            }
        }{乙炔}
    \end{center}\vspace{10pt}
    乙炔可以通过电石和水的反应制取:
    \begin{center}
        \ce{CaC2 + 2H2O -> Ca(OH)2 + C2H4 ^}\\[3mm]
    \end{center}
    乙炔可以在空气中燃烧,火焰明亮伴有黑烟:
    \begin{center}
        \ce{2C2H2 + 5O2 ->T[点燃] 4CO2 + 2H2O}\\[3mm]
    \end{center}
    氧炔焰指的是乙炔在氧气中燃烧的火焰。\\[3mm]
    氧炔焰的温度可以达到3000$^\circ$C,常用于金属的切割和焊接。\\[6mm]
    有机物的加成反应:有机物分子中不饱和碳原子通过断开部分键与其他原子团直接结合的反应。\\[3mm]
    乙炔可以和溴水发生加成反应:\vspace{5pt}
    \begin{center}
        \chemnameinit{\chemfig{C(-[2]H)(-[4]Br)(-[6]H)-C(-[2]H)(-[0]Br)(-[6]H)}}
        \setchemfig{atom sep=15pt}
        \setchemfig{fixed length=true}
        \setchemfig{arrow coeff=0.8}

        \schemestart
            \chemfig{H-C~C-H}
            \+{12pt,10pt,1pt}
            \ce{Br2}
            ~~~~\arrow(foo.mid east--bar.mid west){->}~~~~
            \chemname[5pt]{\chemfig{C(-[6]Br)(-[4]H)=C(-[6]Br)(-[0]H)}}{1,2-二溴乙烯}
        \schemestop
    \end{center}\vspace{10pt}
    乙炔可以通过和氢气的加成反应生成乙烯:\vspace{5pt}
    \begin{center}
        \chemnameinit{\chemfig{C(-[2]H)(-[4]Br)(-[6]H)-C(-[2]H)(-[0]Br)(-[6]H)}}
        \setchemfig{atom sep=15pt}
        \setchemfig{fixed length=true}
        \setchemfig{arrow coeff=0.8}

        \schemestart
            \chemname[5pt]{\chemfig{H-C~C-H}}{乙炔}
            \+{12pt,10pt,1pt}
            \ce{H2}
            ~~~~~\arrow(foo.mid east--bar.mid west){->[\footnotesize 催化剂][\footnotesize 加热]}~~~~
            \chemname[5pt]{\chemfig{C(-[4]H)(-[6]H)=C(-[6]H)(-[0]H)}}{乙烯}
        \schemestop
    \end{center}\vspace{15pt}
    乙炔可以通过和氯化氢的加成反应生成氯乙烯:
    \begin{center}
        \chemnameinit{\chemfig{C(-[4]H)(-[6]H)=C(-[0]H)(-[6]Cl)}}
        \setchemfig{atom sep=15pt}
        \setchemfig{fixed length=true}
        \setchemfig{arrow coeff=0.8}

        \schemestart
            \chemname[5pt]{\chemfig{H-C~C-H}}{乙炔}
            \+{12pt,10pt,1pt}
            \ce{HCl}
            ~~~\arrow(foo.mid east--bar.mid west){->[\footnotesize 催化剂][\footnotesize 加热]}~~~~
            \chemname[5pt]{\chemfig{C(-[4]H)(-[6]H)=C(-[0]H)(-[6]Cl)}}{氯乙烯}
        \schemestop
    \end{center}

\newpage

    有机物的聚合反应:通过相对分子量较小的有机物的互相结合,形成相对分子量很大的有机物。\\[3mm]
    氯乙烯可以通过聚合反应生成聚氯乙烯:
    \begin{center}
        \setchemfig{atom sep=15pt}
        \setchemfig{fixed length=true}
        \setchemfig{arrow coeff=0.8}

        \schemestart
            \chemfig{nCH_2=CH-Cl}
            ~~~~~\arrow(foo.mid east--bar.mid west){->[\footnotesize 催化剂]}~~~~
            \chemfig{\text{\big[}-CH_2-CH_2(-[6]Cl)-\text{\big]$_n$}}
        \schemestop
    \end{center}

\subsubsection{炔烃}
    炔烃指的是一类碳原子间存在一个碳碳叁键的链状碳氢化合物。\\[3mm]
    炔烃的分子式可以使用通式\ce{C_{n}H_{2n-2}}表示。\\[3mm]
    以下列出了一些常见的烯烃:\vspace{5pt}
    \begin{center}
        \chemnameinit{\chemfig{C(-[4]H)(-[6]H)=C(-[6]H)(-[0]H)}}
        \setchemfig{atom sep=15pt}
        \setchemfig{fixed length=true}
        \chemname[5pt]
        {
            \chemfig
            {
                H-C~C-H
            }
        }{乙炔}\qquad\qquad
        \chemname[5pt]
        {
            \chemfig
            {
                H-C~C-C(-[2]H)(-[6]H)-H
            }
        }{丙炔}
    \end{center}\vspace{10pt}
    炔烃的系统命名法:\\[3mm]
    在主链的选择时,还要首先保证选择的主链中含有叁键。\\[3mm]
    在主链的编号时,还要首先保证编号的起点离叁键最近。\\[3mm]
    设主链的碳原子数量为X,设叁键连接的两个碳原子中较小的编号为B,将主链命名为:
    \begin{center}
        B-X炔~~~~~~~~X\,$\in\big\{\text{甲},\text{乙},\text{丙},\text{丁},\cdots\big\}$~~~~~~~~$\text{B}\in\big\{1,2,3,4,\cdots\big\}$\\[5mm]
    \end{center}
    在完成命名时,需要在主链的名称前添加减号。\\[8mm]
    以下列出了部分使用系统命名法的炔烃:\vspace{5pt}
    \begin{center}
        \chemnameinit{\chemfig{CH_2=CH-CH_2=CH_3}}
        \setchemfig{atom sep=15pt}
        \setchemfig{fixed length=true}
        \chemname[15pt]
        {
            \chemfig
            {
                CH~C-CH_2-CH_3
            }
        }{1-丁炔}\qquad\qquad
        \chemname[15pt]
        {
            \chemfig
            {
                CH_3-C~C-CH_3
            }
        }{2-丁炔}
    \end{center}\vspace{20pt}
    \begin{center}
        \chemnameinit{\chemfig{CH_3-CH(-[6]CH_3)-CH_2-CH(-[6]CH_3)-CH_2-CH(-[6]CH_2-[6]CH_3)=CH_2}}
        \setchemfig{atom sep=15pt}
        \setchemfig{fixed length=true}
        \chemname[5pt]
        {
            \chemfig
            {
                CH_3-CH_2-CH(-[6]CH_3)-CH_2-CH(-[6]CH_2-[6]CH_3)-CH-C~CH_2
            }
        }{6-甲基-4-乙基-1-辛炔}
    \end{center}

\end{document}
